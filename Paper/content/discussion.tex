\section{Discussion}\label{sec:discussion}

The results obtained show both the attitude and simulated position response of the quadcopter. 

It is seen that the controllers track the references even though the network delay and the sampling rate affects their performance. The network limits the designed bandwidths, especially for the attitude controller. This is due to the limited frequency in which the sensor data is obtained from the motion tracking system. A faster response is achievable if on board sensors for acquiring the attitude are utilized.

Experimental results could not be presented for the translational controllers, as it has not been possible to implement them with success in due time. The design is however deemed reasonable, as simulations show that the design should work in reality. The attitude controller is implemented and achieves the given references.

%Experimental results could not be presented for the translational controllers. As the attitude design has been validated with the same simulation and it works in reality, it can be suggested that it is not the design but the implementation of the translational controllers which, at this point in the design process prevents the realization of a flying quadcopter.
%
%Simple model
%Translational 
%The attitude controller tracks the given references in roll and pitch. The network delay 
%It is also worth observing how the attitude controller shows a permanent error with respect to the reference. This is generated as a result of the integral controller design because it assumes a constant reference applied to it. This issue, though, does not affect the final position of the quadcopter.
%\fxnote{why the controllers are slow?, the delay is high}