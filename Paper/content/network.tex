\section{Network}\label{sec:network}
The wireless network between the sensor and the quadcopter influences the performance of the controller. This influence is modeled by considering two effects: the delay and the loss of packets in the communication channel.

%The effects on the control performance of the usage of the wireless network to get the sensor data can be divided in two. These are the delay and the packet loss.

The theoretical modeling of these influences has been studied by several researchers in order to obtain a criteria for finding maximum allowable delay and packet loss for the control system to remain stable  \cite{ling}, \cite{nirupam}. However, these approaches often lead to an increased complexity as the network effect is included in the model of the system.

Stability of the system when influenced by the wireless network is instead analyzed using the TrueTime Simulator \cite{TrueTimeNew}. It  provides the option to simulate the network model, the controller design and the system model together. This approach makes it possible to design the controllers taking into account the network effects and, thus, ensure that stability is achieved.

The delay is modeled in the network simulation as constant for all samples. Its value is calculated by adding two time intervals. The first is the time needed for the transmission of the data, that is, the time elapsed since the data is acquired until it is available for the controller, this is a fixed delay formed by a combination of transmission and code execution times. The second is the maximum time elapsed until the controller uses the information and it is estimated as the sampling time minus the execution time of the control loop. With this delay model, the worst case scenario is considered.

The packet loss, defined as a constant probability of loosing a packet, is found experimentally by sending a large amount of packets and examining how many of these are available for the controller. In the experiment, it has been found that more packets are received than control loops are executed, that is, the most recent packet is always available for the controller and, therefore, the packet loss probability is zero.