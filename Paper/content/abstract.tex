% As a general rule, do not put math, special symbols or citations
% in the abstract
\begin{abstract}
Quadcopters are becoming increasingly interesting due to the great variety of usage. A design that is able to make the quadcopter hover and move to a desired position is presented. The system’s coupled behavior and instability raises a challenging control task. This task is solved by implementing a controller design, which is based upon a model that is derived by first principle modelling. This is later linearized since it is desired to use linear controllers. The system is divided into an attitude and translational control loops. These are designed as state space and classical control, respectively. The prototype gets its attitude and position from a motion tracking system, keeping the control in a micro processor on the quadcopter. This layout constitutes a distributed system, where network issues, such as delays and packet losses, are taken into account.
\end{abstract}