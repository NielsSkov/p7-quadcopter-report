% As a general rule, do not put math, special symbols or citations
% in the abstract
\begin{abstract}
Quadcopters are becoming increasingly interesting due to the great variety of usage. In this paper, a design that is able to make the quadcopter hover and move to a desired position is presented. The system’s coupled behavior and instability raises a challenging control task. This task is solved by implementing a controller design, that is based upon a model that is derived by first principle physics. This is later linearized using the Taylor approximation, since it is desired to use a linear approach in the controllers. The total system is made up of multiple subsystems. An attitude and a translational controller are designed as state space control and classical control respectively. The prototype does not carry on board sensors, but gets its position and orientation from a remote sensor, keeping the control in a micro processor on the quadcopter. This layout constitutes a distributed system, where network issues such delays and packet losses need to be taken into account.
\end{abstract}