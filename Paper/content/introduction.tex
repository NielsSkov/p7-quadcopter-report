\section{Introduction}
% no \IEEEPARstart
%Present topic - uses of drones in reality context, chosen because it is a control challenge, rather than revolutionary.
In the last years, the interest for quadcopters has increased due to the great possibilities they offer. Among these, the most well-known ones are surveillance, inspection of big structures and search and rescue missions in difficult environments \cite{droneuses}.

The quadcopter constitutes a control challenge due to its unstable nature and coupled behavior. The system has 6 degrees of freedom, the 3 position coordinates and the 3 orientations, and there are only four actuation variables which are the motor velocities. The dimension of the problem is explained by McKerrow in \cite{draganflyer}.

%Previous Approaches - examples of what others have done to obtain similar goals of stabilization like we pursue. What have others done differently than we plan to do to obtain the same end result.
The control of a quadcopter has been addressed many times in the recent years. In Mian et al. \cite{backstepping} the quadcopter is controlled using a back-stepping technique and non-linear controllers. Other way of solving the issue is presented in Tayebi et al. \cite{quaternionsPD} in which the quadcopter attitude is modeled using quaternions and controlled with a PD based controller. In \cite{MianWang}, Mian and Yang model the system using its dynamic equations and use non linear controllers to achieve a steady flight while in Mokhtari et al. \cite{GHinf} the system is controlled by a mixture of a robust feedback linearizion and a linear GH$_{\infty}$.

%Describe our approach shortly.
The approach presented here models the quadcopter by a first principles method. This approach yields a non linear model that describes the attitude and translational behavior of the quadcopter. The model is then linearized around an equilibrium point, which is chosen to be in hovering position. With the linearized equations, controllers for attitude and translational behaviors are designed. The angular controller is obtained by means of a State Space representation while the translational controller is designed using classical control techniques. In the control system, the translational constitutes an outer loop and sets the reference for the attitude controller.

In the last part of the paper, the simulations and experimental results of the designed controllers are shown and discussed.