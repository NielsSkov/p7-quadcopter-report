\section{Introduction}
% no \IEEEPARstart
%Present topic - uses of drones in reality context, chosen because it is a control challenge, rather than revolutionary.
In the last years, the interest for quadcopters has increased due to the great possibilities they offer. Among these, the most well-known ones are surveillance, inspection of big structures and search and rescue missions in difficult environments. \cite{droneuses}

The quadcopter constitutes a control challenge due to its unstable nature and coupled behavior. The system has six degrees of freedom, the three position coordinates and the three orientations, and there are only four actuation variables, namely the motor rotational speeds \cite{draganflyer}.

%Previous Approaches - examples of what others have done to obtain similar goals of stabilization like we pursue. What have others done differently than we plan to do to obtain the same end result.
The control of a quadcopter has been addressed many times in the recent years. In Mian et al. \cite{backstepping} the quadcopter is controlled using a back-stepping technique and non-linear controllers. Another way of solving the issue is presented in Tayebi et al. \cite{quaternionsPD} in which the quadcopter attitude is modeled using quaternions and controlled with a PD based controller. In \cite{MianWang}, Mian and Wang model the system using its dynamic equations and use non-linear controllers to achieve a steady flight while in Mokhtari et al. \cite{GHinf} the system is controlled by a mixture of a robust feedback linearization and a modified optimization control method.

%Describe our approach shortly.
\fxnote{Why we do not include specific details of our hardware or our network in the methods??}
The approach presented here tries to control the attitude and translational position of the quadcopter using linear controllers and transmitting sensor data through a wireless network. 

In \autoref{sec:model}, the model of the quadcopter is obtained by a first principles method. This approach yields a non-linear model that describes the attitude and translational behavior of the quadcopter. The model is then linearized around an equilibrium point, which is chosen to be in hovering position. 
%
With the linearized equations, controllers for attitude and translational behaviors are designed in \autoref{sec:control}. The attitude controller is obtained by means of a state space representation while the translational controller is designed using classical control. In the control system, the translational constitutes an outer loop and sets the reference for the attitude controller.
%
Since the sensors are not placed in the quadcopter and the information comes from an external motion tracking system \cite{vicon}, an analysis on how the network affects the control loop is also presented, this is done in \autoref{sec:network}.
%
In \autoref{sec:results}, the simulations and experimental results of the designed controllers are presented. They are after discussed in \autoref{sec:discussion}. Lastly, a conclusion is presented in \autoref{sec:conclusion} and possible future work is mentioned in \autoref{sec:futurework}. \fxnote{Are we to present both simulation and real results?}