\subsection{Translational Controller}
The movements of the quadcopter along the inertial frame directions x, y and z are controlled by the translational controllers. It is decided to structure the controllers as cascade loops. The relation between the controllers are presented in \fxnote{figure}

%The translational control systems for x and y are structured as cascade, where the velocity and position are controlled in the inner and outer loop respectively.

\begin{figure}[H]
	\centering
	\includegraphics[scale=0.07]{figures/cascade_paper.png}
	\caption{Cascade loop for the translational controllers where the inner and outer loop control velocity and position respectively.}
	\label{fig:cascade}
\end{figure}
The vertical position is controlled by the cascade loop for the z axis' velocity and position, obtaining the required sum of motor rotational speeds. 
The x and y controller share similar properties as the output for each are an angle reference, namely $\theta_{ref}$ and $\phi_{ref}$ respectively.
Firstly the x and y controllers are designed similarly followed by an individual design for the z controller.\\ 

The inner loop for the x and y translational controllers are now designed followed by the outer loop.
The model equations derived previously, see Equation \ref{eq:AccelerationEqInertial1} and \ref{eq:AccelerationEqInertial2}, are Laplace transformed and put on transfer function in respect to the inner loop, yielding:
\begin{flalign}
    G_{\dot{x}_I}(s)&=\frac{\dot{x}_I (s)}{\theta (s)}=\frac{-k_{th} (\omega_1 ^2 + \omega_2 ^2 + \omega_3 ^2 + \omega_4 ^2)}{m\ s} \\
    G_{\dot{y}_I}(s)&=\frac{\dot{y}_I (s)}{\phi (s)}=\frac{k_{th}(\omega_1 ^2 + \omega_2 ^2 + \omega_3 ^2 + \omega_4 ^2)}{m\ s} 
\end{flalign}

\begin{where}
\va{G_{x_I}}{is the plant for the translational velocity in $x_I$ direction}{1}
\va{G_{y_I}}{is the plant for the translational velocity in $y_I$ direction}{1}
\end{where}
The plants are similar but with different signs. The controller design is carried out for the x translational velocity and applied to the y translational velocity afterwards.\\
A proportional controller is sufficient as the plant already has an integrator, that will eliminate a steady state error and output disturbances. The gain will be the same for both controllers, but must be negative for the x translational controller in order to compensate for its negative plant as this will otherwise be unstable in the closed loop.

The final proportional controllers are as follows:
\begin{align}
C_{\dot{x}I(s)= -0.19}\\
C_{\dot{y}I(s)= 0.19}
\end{align}
\begin{where}
    \va{G_{x_I}}{is the plant for the translational position in $x_I$ direction}{1}
    \va{G_{y_I}}{is the plant for the translational position in $y_I$ direction}{1}
\end{where}
The gain is designed such that it encounters a bandwidth that is three times lower than the attitude model to ensure minimum effect of disturbances.
The plant of the outer loop is simply an integrator to transform velocity to position. The controller of the outer loop is a proportional controller. The outer loop is designed to have three times less bandwidth than the inner loop to ensure minimization of disturbances to secure a stable system.
The proportional controllers are the following:
\begin{align}
C_{\dot{x}O(s)= XX}\\
C_{\dot{y}0(s)= YY}
\end{align}
WHEREs for the controllers are not needed i think - take up too much space. 
The translational z controller is designed as cascade as well. The 

