\documentclass[conference]{IEEEtran}

% Make latex understand and use the typographic
% rules of the language used in the document.
\usepackage[english]{babel}

% Choose the font encoding
\usepackage[T1]{fontenc}

% Use colour in tables
\usepackage[table]{xcolor}
\usepackage{array}
\usepackage{multirow}

% load a colour package
\usepackage{xcolor}
\definecolor{aaublue}{RGB}{33,26,82}% dark blue

% The standard graphics inclusion package
\definecolor{white}{RGB}{255,255,255} % define color white
\usepackage{graphicx}

% Set up how figure and table captions are displayed
\usepackage{caption}
\captionsetup{
  font=footnotesize,% set font size to footnotesize
  labelfont=bf % bold label (e.g., Figure 3.2) font
}

% Enable row combination in tables
\usepackage{multirow}

% Make space between table lines and text
\renewcommand{\arraystretch}{1.5}

% Make the standard latex tables look so much better
\usepackage{array,booktabs}



%%%%%%%%%%%%%%%%%%%%%%%%%%%%%%%%%%%%%%%%%%%%%%%%
% Mathematics
%%%%%%%%%%%%%%%%%%%%%%%%%%%%%%%%%%%%%%%%%%%%%%%%
% Defines new environments such as equation,
% align and split 
\usepackage{amsmath}
\usepackage{relsize}
% Adds new math symbols
\usepackage{amssymb}
% Use theorems in your document
% The ntheorem package is also used for the example environment
% When using thmmarks, amsmath must be an option as well. Otherwise \eqref doesn't work anymore.
\usepackage[framed,amsmath,thmmarks]{ntheorem}
\usepackage{cancel}

%%%%%%%%%%%%%%%%%%%%%%%%%%%%%%%%%%%%%%%%%%%%%%%%
% Page Layout
%%%%%%%%%%%%%%%%%%%%%%%%%%%%%%%%%%%%%%%%%%%%%%%%


%%%%%%%%%%%%%%%%%%%%%%%%%%%%%%%%%%%%%%%%%%%%%%%%
% Bibliography
%%%%%%%%%%%%%%%%%%%%%%%%%%%%%%%%%%%%%%%%%%%%%%%%
%%setting references (using numbers) and supporting i.a. Chicargo-style:
\usepackage{etex}
%\usepackage{etoolbox}
%\usepackage{keyval}
%\usepackage{ifthen}
%\usepackage{url}
%\usepackage{csquotes}
%\usepackage[backend=biber, url=true, doi=true, style=numeric, sorting=none]{biblatex}
%\addbibresource{setup/bibliography.bib}

%%%%%%%%%%%%%%%%%%%%%%%%%%%%%%%%%%%%%%%%%%%%%%%%
% Misc
%%%%%%%%%%%%%%%%%%%%%%%%%%%%%%%%%%%%%%%%%%%%%%%%

%%% Enables the use FiXme refferences. Syntax: \fxnote{...} %%%
\usepackage[footnote, draft, english, silent, nomargin]{fixme}
%With "final" instead of "draft" an error will ocure for every FiXme under compilation.

%%% allows use of lorem ipsum (generate i.e. pagagraph 1 to 5 with \lipsum[1-5]) %%%
\usepackage{lipsum}

%%% Enables figures with text wrapped tightly around it %%%
\usepackage{wrapfig}

\usepackage{float}
\usepackage{caption}
\usepackage{subcaption}
\usepackage{siunitx}
\sisetup{decimalsymbol=comma}
\sisetup{detect-weight}

\usepackage{enumitem}
%\usepackage[citestyle=authoryear,natbib=true]{biblatex}

% Figures - TIKZ
\usepackage{tikz}
\usetikzlibrary{shapes,arrows}
\usepackage[americanresistors,americaninductors,americancurrents, americanvoltages]{circuitikz}

% Wall of text logo
\newcommand{\walloftextalert}[0]{\includegraphics[width=\textwidth]{walloftext.png}}

\usepackage{pdfpages}
\usepackage{lastpage}
\usepackage{epstopdf}

\setlength{\headheight}{21pt}

\hfuzz=\maxdimen
\tolerance = 10000
\hbadness  = 10000

\usepackage{siunitx}
\graphicspath{{./figures/}}