%%%%%%%%%%%%%%%%%%%%%%%%%%%%%%%%%%%%%%%%%%%%%%%%%%%%%
%             UNITS, EQUATIONS AND TEXT             %
%%%%%%%%%%%%%%%%%%%%%%%%%%%%%%%%%%%%%%%%%%%%%%%%%%%%%
%Units:
\newcommand{\unit}[1]{&& \left[\si{#1}\right]} %\newcommand{\unit}[1]{[\si{#1}]}             <<| Use these if you want uations to be
\newcommand{\unitWh}[1]{[\si{#1}]}             %\newcommand{\eq}[2]{&&\si{#1} &= \si{#2}&&}  <<| centered.. .. will appear scrambled
\newcommand{\numUnit}[1]{\ \si{#1}&}           %                                               | from one equation to the next though..
%Equation:                                     %                                               | and does not work with long equations.. :/
\newcommand{\eq}[2]{\si{#1} &= \si{#2}}
\newcommand{\arw}{&& &\Updownarrow&&}
\newcommand{\eqOne}[2]{\si{#1} &= \si{#2} &\nonumber\\}
\newcommand{\eqTwo}[1]{&\ \ \ \ \si{#1}& \nonumber\\}
\newcommand{\eqThree}[1]{&\ \ \ \ \si{#1}&}
%Text:
\newcommand{\tx}[1]{\text{#1}}
%Vectors
\renewcommand{\vec}[1]{\boldsymbol{\mathbf{#1}}}
%Vertical line in equations ie. |_x=y (whereTwo stacks two equalities at the line)
\newcommand{\lineWhere}[1]{ \left.\rule{0cm}{.5cm}\right\vert\rule{0cm}{.4cm}_{\substack{\rule{0cm}{.15cm}\\ \si{#1} }} }
\newcommand{\lineWhereTwo}[2]{ \left.\rule{0cm}{.67cm}\right\vert\rule{0cm}{.5cm}_{\substack{\si{#1} \rule{0cm}{.19cm}\\\vspace{-.1cm}\\ \si{#2}}} }

%%%%%%%%%%%%%%%%%%%%%%%%%%%%%%%%%%%%%%%%%%%%%%%%%%%%%
%                 TIKZ SETTINGS                     %
%%%%%%%%%%%%%%%%%%%%%%%%%%%%%%%%%%%%%%%%%%%%%%%%%%%%%
%\usetikzlibrary{arrows.meta}
\tikzset{
  block/.style    = {draw, thick, rectangle,
                     minimum height = 2.1em,
                     minimum width = 1.7em},
  sum/.style      = {draw, circle, inner sep=1.5pt},
}

%%%%%%%%%%%%%%%%%%%%%%%%%%%%%%%%%%%%%%%%%%%%%%%%%%%%%
%               WHERE, AFTER MATH                   %
%%%%%%%%%%%%%%%%%%%%%%%%%%%%%%%%%%%%%%%%%%%%%%%%%%%%%
\usepackage{xifthen}

\newenvironment{where}{Where:\\}{\\}
\newcommand{\va}[3]
{
  \begin{tabular}{p{10pt} p{145pt} l}
    { $#1$ } & { #2 } & \ifthenelse{\isempty{ #3 }}  {}  {[{\si{#3}}]} \\
  \end{tabular}\\
}