\chapter{Implementation}




\section{Scheduler}
\begin{lstlisting}[style=customcpp,
                    caption={Code for initialization, creation of the different tasks, start sequence for the motors and call to the scheduler.}, 
                    label=lst:scheduler]
int main()
{
    // Initialization
    PWM_init(0);
    USART_Init(MYUBRR);
    ADC_Init();
    
    // Task Creation
    xTaskCreate(Controllers, "Control", 1000, NULL, configMAX_PRIORITIES - 1, NULL );
    xTaskCreate(Comunication, "Com", 1000, NULL, configMAX_PRIORITIES - 2, &xHandle);
    
    // Star sequence for the motor controllers
    _delay_ms(1000);
    int duty = 128;
    Set_PWM_duty(duty, duty, duty, duty);
    _delay_ms(10000);
 
    // Scheduler Start
    vTaskStartScheduler();
    return 0;
}                    
\end{lstlisting}


\section{Communication}
\begin{lstlisting}[style=customcpp,
                caption={Code for the comunication task.}, 
                label=lst:scheduler]
void Comunication(void *pvParameters)
{
    while (1)
    {
        int pack = 0;
        pack = CheckPackageArrival();
        if (pack)
            GetPackage();
    }
    vTaskDelete(NULL);
}
\end{lstlisting}


\section{Controllers}
\begin{lstlisting}[style=customcpp,
caption={Code for the controller task.}, 
label=lst:scheduler]
void Controllers(void *pvParameters)
{
    portTickType xLastWakeTime;
    xLastWakeTime = xTaskGetTickCount();
    
    while (1)
    {
        Controller();
        count++;
        ApplyVelocities();
        if (reading)
        {
            vTaskDelete(xHandle);
            xTaskCreate(Comunication, "Com", 1000, NULL, configMAX_PRIORITIES - 2, &xHandle);
            reading = 0;
        }
        vTaskDelayUntil(&xLastWakeTime, 35);
    }
    vTaskDelete(NULL);
}
\end{lstlisting}


\begin{lstlisting}[style=customcpp,
caption={Code for the controllers.}, 
label=lst:scheduler]
    ...
    vel_ref_k[i] = -208.8*pos_e_k[i] + 198.2*pos_e_k1[i] + vel_ref_k1[i];
    ...
    xint_k[i] = T / 2 * (e_k[i] + e_k1[i]) + xint_k1[i];
    ...
    oint_k[i] = T / 2 * (o_k[i] + o_k1[i]) + oint_k1[i];
    ...

\end{lstlisting}