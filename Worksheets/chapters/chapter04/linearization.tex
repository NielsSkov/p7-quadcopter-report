\section{Linearization} \label{sec:Linearization}
%Since the goal is to design a linear controller, it is necessary to linearize the equations around an equilibrium point.
%
%The attitude equations in the body frame, \fxnote{ref needed}, must be linearized since they contain squared velocities. The translational equations in the inertial frame, \autoref{eq:AccelerationEqInertial1}, \ref{eq:AccelerationEqInertial2}, and \ref{eq:AccelerationEqInertial3}, must also be linearized since these contain trigonometric functions and squared velocities.
%
%The equilibrium point is chosen to be where the quadcopter is hovering at a fixed position in the inertial frame. This happens when roll and pitch are kept to zero. The forces along the x and y axes in some cases also depend on yaw, however, as the forces were transformed from the body frame to the inertial frame, as described in \fxnote{ref needed}, in the order roll, pitch and then yaw, the yaw will not affect the translational forces, and so, it is not, as such, part of the equilibrium point.
%
%The equilibrium point for the four angular velocities of the motors is again that which allows the quadcopter to hover, that is, the four equal angular velocities which keeps a constant z-coordinate in the inertial frame.
%The linearization is done for each equation using the first order Taylor approximation.
Designing a linear controller requires linearization of the model equations around an equilibrium point. This is done for each equation using the first order Taylor approximation, see \autoref{taylor}.
%
\begin{flalign}
	f(x) &\approx f(\overline{x}) + f'(\overline{x}) (x-\overline{x})  \rightarrow\ \Delta f(x) \approx f'(\overline{x}) \Delta x
	\label{taylor}
\end{flalign}
In this equation, $\overline{x}$ represents the linearization point.

To apply the approximation, the function to be approximated must be differentiated with respect to each of the present variables.
\begin{flalign}
	f &= f(x_1,x_2,...,x_n) \nonumber \\
	\Delta f&=\frac{\partial f}{\partial x_1}\bigg|_{\overline{x}_1,\overline{x}_2,...,\overline{x}_n}\ \Delta x_1 + \frac{\partial f}{\partial x_2}\bigg|_{\overline{x}_1,\overline{x}_2,...,\overline{x}_n}\ 
	\Delta x_{2}+...+ \frac{\partial f}{\partial x_n}\bigg|_{\overline{x}_1,\overline{x}_2,...,\overline{x_n}}\ \Delta x_n
	\label{eq:dummytaylor}
\end{flalign}

Once linearized, the function is expressed in terms of variations from the linearization point. This is represented by the symbol $\Delta$ in all linearized equations.

In the quadcopter model, the attitude equations, \autoref{eq:AngleEqVelocities1}, \ref{eq:AngleEqVelocities2} and \ref{eq:AngleEqVelocities3}, and the translational equations, \autoref{eq:AccelerationEqInertial1}, \ref{eq:AccelerationEqInertial2}, and \ref{eq:AccelerationEqInertial3}, must be linearized since these contain trigonometric functions and squared velocities. 

The linearization point is chosen to be where all the angular and translational velocities and accelerations are equal to zero. In \autoref{eq:AccelerationEqInertialVelocities1} and \ref{eq:AccelerationEqInertialVelocities2} it is seen that to obtain a zero in $x_{\mathrm{I}}$ and $y_{\mathrm{I}}$ accelerations, the pitch and roll of the quadcopter should be zero as well. To keep the acceleration in the $z_{\mathrm{I}}$ axis in the inertial frame equal to zero, the rotational speeds of the motors, need to be calculated. This can be done from \autoref{eq:AccelerationEqInertialVelocities3}.
%
\begin{flalign}
	m \overline{\ddot{z}}_{\mathrm{I}} &= F_g-k_{\mathrm{th}} ({\overline{\omega}_1}^2+{\overline{\omega}_2}^2+{\overline{\omega}_3}^2+{\overline{\omega}_4}^2) \cos\overline{\phi} \cos\overline{\theta} \label{eq:equilibriumomegas}
\end{flalign}
%
Since all $\overline{\ddot{z}_I}$, $\overline{\phi}$ and $\overline{\theta}$ are equal to zero, \autoref{eq:equilibriumomegas} can be rearranged to \autoref{eq:equilibriumomegas2}.
%
\begin{flalign}
	\overline{\omega}_i=\sqrt{\frac{m g}{4k_{th}}}
	\label{eq:equilibriumomegas2}
\end{flalign}
%
Applying the linearization procedure to the attitude model equations around the linearization point, \autoref{eqAngleLin1}, \ref{eqAngleLin2} and \ref{eqAngleLin3} are obtained.
\begin{flalign}
  J_x \Delta\ddot{\phi}   &= 2 k_{\mathrm{th}} L {\overline{\omega}_4} \Delta \omega_4 - 2 k_{\mathrm{th}} L {\overline{\omega}_2} \Delta \omega_2
  \label{eqAngleLin1} \\
  J_y \Delta\ddot{\theta} &= 2 k_{\mathrm{th}} L \overline{\omega}_1 \Delta \omega_1 - 2 k_{\mathrm{th}} L \overline{\omega}_3 \Delta \omega_3
  \label{eqAngleLin2} \\
  J_z \Delta\ddot{\psi}   &= 2 k_{\mathrm{d}} {\overline{\omega}_1} \Delta \omega_1 - 2 k_{\mathrm{d}} {\overline{\omega}_2} \Delta \omega_2 + 2 k_{\mathrm{d}} {\overline{\omega}_3} \Delta \omega_3 - 2 k_{\mathrm{d}} {\overline{\omega}_4} \Delta \omega_4 \label{eqAngleLin3}
\end{flalign} 
%
\begin{where}
  \va{ \Delta\ddot{\phi}     } {is the change in roll angular acceleration from equilibrium}         { rad \  s^{-2} }
  \va{ \Delta\ddot{\theta}   } {is the change in pitch angular acceleration from equilibrium}        { rad \  s^{-2} }
  \va{ \Delta\ddot{\psi}     } {is the change in yaw angular acceleration from equilibrium}          { rad \  s^{-2} }
  \va{ \overline{\omega}_i } {is the angular velocity of each motor in equilibrium}             { rad \  s^{-1} }
  \va{ \Delta \omega_i       } {is the change in angular velocity from equilibrium of each motor} { rad \  s^{-1} }
\end{where}

In a similar way as with the attitude equations, the translational model can be linearized, leading to \autoref{eq:TransLinearEquations1}, \ref{eq:TransLinearEquations2} and \ref{eq:TransLinearEquations3}.
%
\begin{flalign}
  m \Delta\ddot{x}_{\mathrm{I}} &= -k_{\mathrm{th}} ({\overline{\omega}_1}^2+{\overline{\omega}_2}^2+{\overline{\omega}_3}^2+{\overline{\omega}_4}^2)  \Delta\theta \label{eq:TransLinearEquations1} \\
  m \Delta\ddot{y}_{\mathrm{I}} &=  k_{\mathrm{th}} ({\overline{\omega}_1}^2+{\overline{\omega}_2}^2+{\overline{\omega}_3}^2+{\overline{\omega}_4}^2) \Delta\phi \label{eq:TransLinearEquations2}\\
  m \Delta\ddot{z}_{\mathrm{I}} &= -2 k_{\mathrm{th}} \overline{\omega}_1 \Delta\omega_1 -2 k_{\mathrm{th}} \overline{\omega}_2 \Delta\omega_2 -2 k_{\mathrm{th}} \overline{\omega}_3 \Delta\omega_3-2 k_{\mathrm{th}} \overline{\omega}_4 \Delta\omega_4 \label{eq:TransLinearEquations3}
\end{flalign} 
%
\begin{where}
  \va{\Delta\ddot{x_{\mathrm{I}}}  }{ is the change in linear acceleration from equilibrium in $x_{\mathrm{I}}$ direction }{ m \  s^{-2}}
  \va{\Delta\ddot{y_I}  }{ is the change in linear acceleration from equilibrium in $y_{\mathrm{I}}$ direction }{ m \  s^{-2} }
  \va{\Delta\ddot{z_I}  }{ is the change in linear acceleration from equilibrium in $z_{\mathrm{I}}$ direction }{ m \  s^{-2} }
  \va{\Delta \phi       }{ is the change in roll from equilibrium                          }{ rad            }
  \va{\Delta \theta     }{ is the change in pitch from equilibrium                         }{ rad            }
  \va{\Delta \psi       }{ is the change in yaw from equilibrium                           }{ rad            }
  \va{\overline{\phi}   }{ is the roll in equilibrium                                      }{ rad            }
  \va{\overline{\theta} }{ is the pitch in equilibrium                                     }{ rad            }
  \va{\overline{\psi}   }{ is the yaw in equilibrium                                       }{ rad            }
\end{where}
