\section{Linearization} \label{sec:Linearization}
%Since the goal is to design a linear controller, it is necessary to linearize the equations around an equilibrium point.
%
%The attitude equations in the body frame, \fxnote{ref needed}, must be linearized since they contain squared velocities. The translational equations in the inertial frame, \autoref{eq:AccelerationEqInertial1}, \ref{eq:AccelerationEqInertial2}, and \ref{eq:AccelerationEqInertial3}, must also be linearized since these contain trigonometric functions and squared velocities.
%
%The equilibrium point is chosen to be where the quadcopter is hovering at a fixed position in the inertial frame. This happens when roll and pitch are kept to zero. The forces along the x and y axes in some cases also depend on yaw, however, as the forces were transformed from the body frame to the inertial frame, as described in \fxnote{ref needed}, in the order roll, pitch and then yaw, the yaw will not affect the translational forces, and so, it is not, as such, part of the equilibrium point.
%
%The equilibrium point for the four angular velocities of the motors is again that which allows the quadcopter to hover, that is, the four equal angular velocities which keeps a constant z-coordinate in the inertial frame.
%The linearization is done for each equation using the first order Taylor approximation.
The goal of designing a linear controller requires a linearization of the model equations around an equilibrium point. This is done for each equation using the first order Taylor approximation. See \autoref{taylor}.
%
\begin{flalign}
	f(x) &\approx f(x_0) + f'(x_0) (x-x_0)
	\label{taylor}
\end{flalign}

The attitude equations, \autoref{eq:AngleEqVelocities1}, \ref{eq:AngleEqVelocities2} and \ref{eq:AngleEqVelocities3}, and the translational equations, \autoref{eq:AccelerationEqInertial1}, \ref{eq:AccelerationEqInertial2}, and \ref{eq:AccelerationEqInertial3}, must be linearized since these contain trigonometric functions and squared velocities. 

The equilibrium point is chosen to be where the quadcopter is hovering at a fixed position in the inertial frame. This choice makes all equilibrium points equal to zero both angular and translational. In \autoref{eq:AccelerationEqInertialVelocities1} and \autoref{eq:AccelerationEqInertialVelocities2} it can be seen that to yield a zero in x and y accelerations, the pitch and roll of the quadcopter should be zero as well. To keep the acceleration in the z axis in the inertial frame equal to zero, the rotational speed of the motor, which ensures the equilibrium points, needs to be calculated. This can be done from \autoref{eq:AccelerationEqInertialVelocities3}.

\begin{flalign}
	m\cdot\overline{\ddot{z}_I} &= F_g-k_{th}\cdot({\overline{\omega_1}}^2+{\overline{\omega_2}}^2+{\overline{\omega_3}}^2+{\overline{\omega_4}}^2)\cdot\cos(\overline{\phi})\cdot\cos(\overline{\theta})
\end{flalign}

Since all the equilibrium points are set to zero the $\overline{\ddot{z}_I}$, $\overline{\phi}$ and $\overline{\theta}$ are equal to zero. The four rotational speeds of the motor should be the same to keep the acceleration in the z axis. Hence it is possible to combine the four to one, $\overline{\omega}_i$. By utilizing these changes the equation can be rearrange to the following.

\begin{flalign}
	\overline{\omega}_i=\sqrt{\frac{m\cdot g}{4k_{th}}}
	\label{eq:equilibriumomegas}
\end{flalign}

Applying the approximation to the attitude model equations around the equilibrium point, \autoref{eqAngleLin1}, \ref{eqAngleLin2} and \ref{eqAngleLin3} are obtained.

\fxnote{A little more explanation on how the delta comes into play  and how it relates to (3.16)}

\begin{flalign}
  J_x\Delta\ddot{\phi}   &= 2k_{th}L({\overline{\omega}_4} \Delta \omega_2-{\overline{\omega}_2} \Delta \omega_4)
  \label{eqAngleLin1} \\
  J_y\Delta\ddot{\theta} &= 2k_{th} L({\overline{\omega}_1} \Delta \omega_1-{\overline{\omega}_3} \Delta \omega_3) 
  \label{eqAngleLin2} \\
  J_z\Delta\ddot{\psi}   &= 2k_d({\overline{\omega}_1}\Delta \omega_1-{\overline{\omega}_2}\Delta \omega_2+{\overline{\omega}_3}\Delta \omega_3-{\overline{\omega}_4} \Delta \omega_4) \label{eqAngleLin3}
\end{flalign} 
%
\begin{where}
  \va{ \Delta\ddot{\phi}     } {is the change in roll angular acceleration from equilibrium}         { rad \cdot s^{-2} }
  \va{ \Delta\ddot{\theta}   } {is the change in pitch angular acceleration from equilibrium}        { rad \cdot s^{-2} }
  \va{ \Delta\ddot{\psi}     } {is the change in yaw angular acceleration from equilibrium}          { rad \cdot s^{-2} }
  \va{ \overline{\omega}_i } {is the angular velocity of each motor in equilibrium}             { rad \cdot s^{-1} }
  \va{ \Delta \omega_i       } {is the change in angular velocity from equilibrium of each motor} { rad \cdot s^{-1} }
\end{where}
%\begin{note}
%\begin{flalign}
%       \eqOne{m\cdot\Delta\ddot{x}_I}{(-k_{th}\cdot(2\textbf{ }{\overline{\omega}_1}^2\cdot\Delta\omega_1+2\textbf{ }{\overline{\omega}_2}^2\cdot\Delta\omega_2+2\textbf{ }{\overline{\omega}_3}^2\cdot\Delta\omega_3+2\textbf{ }{\overline{\omega}_4}^2\cdot\Delta\omega_4)\cdot\sin(\overline{\theta})-}
%       \eqTwo{-k_{th}\cdot({\overline{\omega}_1}^2+{\overline{\omega}_2}^2+{\overline{\omega}_3}^2+{\overline{\omega}_4}^2)\cdot\cos(\overline{\theta})\Delta\theta} &\\
%       \eqOne{m\cdot\Delta\ddot{y}_I}{(k_{th}\cdot(2\textbf{ }{\overline{\omega}_1}^2\cdot\Delta\omega_1+2\textbf{ }{\overline{\omega}_2}^2\cdot\Delta\omega_2+2\textbf{ }{\overline{\omega}_3}^2\cdot\Delta\omega_3+2\textbf{ }{\overline{\omega}_4}^2\cdot\Delta\omega_4)\cdot\sin(\overline{\phi})\cdot\cos(\overline{\theta})-}
%       \eqTwo{-k_{th}\cdot({\overline{\omega}_1}^2+{\overline{\omega}_2}^2+{\overline{\omega}_3}^2+{\overline{\omega}_4}^2)\cdot\sin(\overline{\phi})\cdot\sin(\overline{\theta})\cdot\Delta\theta+}\nonumber\\
%       \eqThree{+k_{th}\cdot({\overline{\omega}_1}^2+{\overline{\omega}_2}^2+{\overline{\omega}_3}^2+{\overline{\omega}_4}^2)\cdot\cos(\overline{\phi})\cdot\cos(\overline{\theta})\cdot\Delta\phi} &\\
%       \eqOne{m\cdot\Delta\ddot{z}_I}{(-k_{th}\cdot(2\textbf{ }{\overline{\omega}_1}^2\cdot\Delta\omega_1+2\textbf{ }{\overline{\omega}_2}^2\cdot\Delta\omega_2+2\textbf{ }{\overline{\omega}_3}^2\cdot\Delta\omega_3+2\textbf{ }{\overline{\omega}_4}^2\cdot\Delta\omega_4)\cdot\cos(\overline{\phi})\cdot\cos(\overline{\theta})-}
%       \eqTwo{-k_{th}\cdot({\overline{\omega}_1}^2+{\overline{\omega}_2}^2+{\overline{\omega}_3}^2+{\overline{\omega}_4}^2)\cdot\sin(\overline{\phi})\cdot\cos(\overline{\theta})\cdot\Delta\theta-}\nonumber\\
%       \eqThree{-k_{th}\cdot({\overline{\omega}_1}^2+{\overline{\omega}_2}^2+{\overline{\omega}_3}^2+{\overline{\omega}_4}^2)\cdot\cos(\overline{\phi})\cdot\sin(\overline{\theta})\cdot\Delta\phi}
%       \label{eq:LinearEquations}
%\end{flalign}
%\end{note}

In a similar way as with the attitude equations, the translational model can be linearized, leading to \autoref{eq:TransLinearEquations1}, \ref{eq:TransLinearEquations2} and \ref{eq:TransLinearEquations3}.
\begin{flalign}
  m\cdot\Delta\ddot{x}_I &= -k_{th}\cdot({\overline{\omega}_1}^2+{\overline{\omega}_2}^2+{\overline{\omega}_3}^2+{\overline{\omega}_4}^2)\cdot\cos(\overline{\theta}) \cdot \Delta\theta \label{eq:TransLinearEquations1} \\
  m\cdot\Delta\ddot{y}_I &=  k_{th}\cdot({\overline{\omega}_1}^2+{\overline{\omega}_2}^2+{\overline{\omega}_3}^2+{\overline{\omega}_4}^2)\cdot\cos(\overline{\phi})\cdot\cos(\overline{\theta})\cdot\Delta\phi \label{eq:TransLinearEquations2}\\
  m\cdot\Delta\ddot{z}_I &= -2\textbf{ }k_{th}\cdot({\overline{\omega}_1}\cdot\Delta\omega_1+{\overline{\omega}_2}\cdot\Delta\omega_2+{\overline{\omega}_3}\cdot\Delta\omega_3+{\overline{\omega}_4}\cdot\Delta\omega_4)\cdot\cos(\overline{\phi})\cdot\cos(\overline{\theta})\label{eq:TransLinearEquations3}
\end{flalign} 
%
\begin{where}
  \va{\Delta\ddot{x_I}  }{ is the change in linear acceleration from equilibrium in $x_I$ direction }{ m \cdot s^{-2} } \\
  \va{\Delta\ddot{y_I}  }{ is the change in linear acceleration from equilibrium in $y_I$ direction }{ m \cdot s^{-2} } \\
  \va{\Delta\ddot{z_I}  }{ is the change in linear acceleration from equilibrium in $z_I$ direction }{ m \cdot s^{-2} } \\
  \va{\Delta \phi       }{ is the change in roll from equilibrium                          }{ rad            } \\
  \va{\Delta \theta     }{ is the change in pitch from equilibrium                         }{ rad            } \\
  \va{\Delta \psi       }{ is the change in yaw from equilibrium                           }{ rad            } \\
  \va{\overline{\phi}   }{ is the roll in equilibrium                                      }{ rad            } \\
  \va{\overline{\theta} }{ is the pitch in equilibrium                                     }{ rad            } \\
  \va{\overline{\psi}   }{ is the yaw in equilibrium                                       }{ rad            }
\end{where}

\fxnote{Hmm, that depends on what you mean. The direction of acceleration in the inertial frame must depend on what direction the quad is tilting, which again depends on the yaw?}

It is worth mentioning that the yaw angle does not affect the translational accelerations in the inertial frame. This occurs due to the way the rotation matrix has been defined. It assumes that rotations are performed by first turning in roll direction, then in pitch direction and finally in yaw direction. This order makes the yaw angle not relevant as the force decomposition in the inertial frame directions is independent of the yaw angle as long as this is assumed to be the last angle that changes.