\section{Translational Model} \label{sec:TranslationalModel}
The aim of this section is to model the translational behavior of the quadcopter. This can be done from \autoref{fig:droneDiagram} in \autoref{sec:AttitudeModel}. An equation for translational accelerations of each specific axis in the inertial frame can be derived by using Newton's Second Law.
%
\begin{flalign}
    m \cdot a = \sum F
\end{flalign}
%
\begin{where}
\va{m}{is the mass}{kg}
\va{a}{is the translational acceleration specific for each axis in the inertial frame}{m \cdot s^{-2}}
\va{F}{are the forces applied to the system}{N}
\end{where}

All the forces that are applied to the quadcopter create an acceleration on it. The forces are the gravity and the thrust that the propellers generate. 

A seen in \autoref{fig:framesDiagram} in \autoref{sec:ModelOverview} the gravitational force has effect only in the \si{z_I} direction and the four forces from the motors need to be transformed into the inertial system. This is done through the rotation matrix showed in \autoref{eq:RotMatrix}.
\fxnote{Anders: Hmmm, you need more details on how you do this? How do you use the 3x3 rotation matrix to obtain (3.10)-(3.12)?}

Taking into account this transformation, the movement can be described using \autoref{eq:AccelerationEqInertial1}, \ref{eq:AccelerationEqInertial2} and \ref{eq:AccelerationEqInertial3}.
%
\begin{flalign}
    m\cdot\ddot{x}_I &= -(F1+F2+F3+F4)\cdot\sin(\theta)  \label{eq:AccelerationEqInertial1}\\
    m\cdot\ddot{y}_I &= -(F1+F2+F3+F4)\cdot(-\sin(\phi))\cdot\cos(\theta)  \label{eq:AccelerationEqInertial2}\\
    m\cdot\ddot{z}_I &= F_g-(F1+F2+F3+F4)\cdot\cos(\phi)\cdot\cos(\theta)
    \label{eq:AccelerationEqInertial3}
\end{flalign}
%
\begin{where}
    \va{\ddot{x}_I} {is the translational acceleration in the $x_I$ direction}        {m \cdot s^{-2} }
    \va{\ddot{y}_I} {is the translational acceleration in the $y_I$ direction}        {m \cdot s^{-2} }
    \va{\ddot{z}_I} {is the translational acceleration in the $z_I$ direction}        {m \cdot s^{-2} }
    \va{F_g} {is the gravitational force acting on the quadcopter} {N}
\end{where}

This equations can also be rewritten taking into account that the forces are assumed to be proportional to the square of the speeds of the motors. This assumption results in \autoref{eq:AccelerationEqInertialVelocities1}, \ref{eq:AccelerationEqInertialVelocities2} and \ref{eq:AccelerationEqInertialVelocities3}.
%
\begin{flalign}
    m\cdot\ddot{x}_I &= -k_{th}\cdot({\omega_1}^2+{\omega_2}^2+{\omega_3}^2+{\omega_4}^2)\cdot\sin(\theta)  \label{eq:AccelerationEqInertialVelocities1}\\
    m\cdot\ddot{y}_I &= -k_{th}\cdot({\omega_1}^2+{\omega_2}^2+{\omega_3}^2+{\omega_4}^2)\cdot(-\sin(\phi))\cdot\cos(\theta) \label{eq:AccelerationEqInertialVelocities2}\\
    m\cdot\ddot{z}_I &= F_g-k_{th}\cdot({\omega_1}^2+{\omega_2}^2+{\omega_3}^2+{\omega_4}^2)\cdot\cos(\phi)\cdot\cos(\theta)
    \label{eq:AccelerationEqInertialVelocities3}
\end{flalign}
%
\autoref{eq:AccelerationEqInertialVelocities1}, \ref{eq:AccelerationEqInertialVelocities2} and \ref{eq:AccelerationEqInertialVelocities3} are the final model expressions of the translational model.

Now that both the attitude model and the translational model are derived, it is desirable to linearise the model expressions. 