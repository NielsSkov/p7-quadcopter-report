\section{Translational Model}
In this section the translational behaviour of the quadcopter is modelled. First the dynamics are descriped in the body frame after which it is translated to the inertial frame.
From the figure in previous section, \fxnote{insert figure reference}, it is possible to derive equations for the linear accelerations in the body frame using Newton's second law.

Theory and so on\\

\subsection{Linear Accelerations - Body Frame}
The linear acceleration components along the body coordinate frame directions evolve according to the equations below. They come from applying Newton's second law. As the gravity needs to be transformed from the inertial frame to the body frame, it has been multiplied by the rotation matrix (\eqref{rotMatrix}) that depends on the roll, pitch and yaw angles.
%
\begin{flalign}
	\eq{m\cdot\ddot{x}_B}{F_g(-\cos(\phi) \sin(\theta) \cos(\psi) + \sin(\phi) \sin (\psi))} &\\
	\eq{m\cdot\ddot{y}_B}{F_g(\cos(\phi) \sin(\theta) \sin(\psi) + \sin(\phi) \cos(\psi))}&\\
	\eq{m\cdot\ddot{z}_B}{-(F_1+F_2+F_3+F_4)+F_g\cdot \cos(\phi)\cos(\theta)}
\label{eq:AccelerationEqBody}
\end{flalign}
%
\hspace{6mm} Where:\\
\begin{tabular}{ p{1cm} l l l}
	& \si{\ddot{x_B}} 	 	& is the linear acceleration in \si{x_B} direction 	&\unitWh{m \cdot s^{-2}} \\
	& \si{\ddot{y_B}} 		& is the linear acceleration in \si{y_B} direction   &\unitWh{m \cdot s^{-2}} \\
	& \si{\ddot{z_B}}	    & is the linear acceleration in \si{x_B} direction     &\unitWh{m \cdot s^{-2}} \\
	& \si{\phi}	 			& is the roll position 	&\unitWh{rad} \\
	& \si{\theta} 			& is the pitch position    &\unitWh{rad} \\
	& \si{\psi}    			& is the yaw position      &\unitWh{rad} \\
	& \si{F_g}    			& is the magnitude of the gravity force      &\unitWh{N} \\
\end{tabular}


As happened with the equations of the angular movement, the forces and torques can be written as a function of the motors' speeds.
\begin{flalign}
	\eq{m\cdot\ddot{x}_B}{F_g(-\cos(\phi) \sin(\theta) \cos(\psi) + \sin(\phi) \sin (\psi))} &\\
	\eq{m\cdot\ddot{y}_B}{F_g(\cos(\phi) \sin(\theta) \sin(\psi) + \sin(\phi) \cos(\psi))}&\\
	\eq{m\cdot\ddot{z}_B}{-k_{th}\cdot(\omega^2_1+\omega^2_2+\omega^2_3+\omega^2_4)+F_g\cdot \cos(\phi)\cos(\theta)}
	\label{eq:AccelerationEqBodyVelocities}
\end{flalign}

The transformation of the gravity force is explained by three consecutive rotations (roll, pitch and yaw) combined in one rotation matrix \si{R_{\phi, \theta, \psi}}.

\begin{minipage}{0.3\linewidth}
	\begin{flalign}
		\si{R_\phi} &=
		\begin{bmatrix}
			\ \si{1}                & \si{0}                & \si{0} \ \ \ \\ 
			\ \si{0}  				& \si{c(\phi)} 		& \si{-s(\phi)}                 \ \ \ \\ 
			\ \si{0}                & \si{s(\phi)}       & \si{c(\phi)}                  \ \ \  
		\end{bmatrix}  \nonumber 
	\end{flalign}
\end{minipage}\hfill
%
\begin{minipage}{0.3\linewidth}
	\begin{flalign}
		\si{R_\theta} &=
		\begin{bmatrix}
			\ \si{c(\theta)}      & \si{0}       & \si{s(\theta)} \ \ \ \\ 
			\ \si{0}  				& \si{1} 	   & \si{0}                 \ \ \ \\ 
			\ \si{-s(\theta)}     & \si{0}       & \si{c(\theta)}                  \ \ \  
		\end{bmatrix}   \nonumber 
	\end{flalign}
\end{minipage}\hfill
%
\begin{minipage}{0.3\linewidth}
	\begin{flalign}
		\si{R_\phi} &=
		\begin{bmatrix}
			\ \si{c(\psi)}                & \si{-s(\psi)}                & \si{0} \ \ \ \\ 
			\ \si{s(\psi)}  				& \si{c(\psi)} 		& \si{0}                 \ \ \ \\ 
			\ \si{0}                & \si{0}       & \si{1}                  \ \ \  
		\end{bmatrix} \nonumber 
	\end{flalign}
\end{minipage}\hfill

\begin{flalign}
	\si{R_{\phi, \theta, \psi}} &=
	\begin{bmatrix}
		\ \si{c(\theta) \cdot c(\psi)}                & \si{-c(\theta) \cdot s(\psi)}  & \si{s(\theta)} \ \ \ \\ 
		\ \si{s(\phi) \cdot s(\theta) \cdot s(\psi) + c(\phi) \cdot s(\psi)}  	  & \si{-s(\phi) \cdot s(\theta) \cdot s(\psi) + c(\phi) \cdot c(\psi)} 		& \si{-s(\phi) \cdot c(\theta)}                 \ \ \ \\ 
		\ \si{-c(\phi) \cdot s(\theta) \cdot c(\psi) + s(\phi) \cdot s(\psi)}  	  & \si{c(\phi) \cdot s(\theta) \cdot s(\psi) + s(\phi) \cdot c(\psi)} 		& \si{c(\phi) \cdot c(\theta)}                 \ \ \ 
	\end{bmatrix} 	\label{rotMatrix}
\end{flalign}


\subsection{Linear Accelerations - Inertial Frame}
It is also possible to explain the linear acceleration in the inertial reference frame. This approach is useful if the aim of the controller is to give references of acceleration, velocity or position in this frame.

In this case the gravity force has effect only in the \si{z_B} direction and the four forces from the motors need to be transformed into the inertial system. This is done using the inverse of the rotation matrix (\eqref{invRotMatrix}).
%
\begin{flalign}
	\eq{m\cdot\ddot{x}_I}{-(F1+F2+F3+F4)\cdot\sin(\theta)} & \label{eq:AccelerationEqInertial1}\\
	\eq{m\cdot\ddot{y}_I}{-(F1+F2+F3+F4)\cdot(-\sin(\phi))\cdot\cos(\theta)} & \label{eq:AccelerationEqInertial2}\\
	\eq{m\cdot\ddot{z}_I}{F_g-(F1+F2+F3+F4)\cdot\cos(\phi)\cdot\cos(\theta)}
	\label{eq:AccelerationEqInertial3}
\end{flalign}
%
\hspace{6mm} Where:\\
\begin{tabular}{ p{1cm} l l l}
	& \si{\ddot{x_I}} 	 	& is the linear acceleration in \si{x_I} direction 	&\unitWh{m \cdot s^{-2}} \\
	& \si{\ddot{y_I}} 		& is the linear acceleration in \si{y_I} direction   &\unitWh{m \cdot s^{-2}} \\
	& \si{\ddot{z_I}}	    & is the linear acceleration in \si{x_I} direction     &\unitWh{m \cdot s^{-2}} \\
\end{tabular}


\begin{flalign}
	\eq{m\cdot\ddot{x}_I}{-k_{th}\cdot({\omega_1}^2+{\omega_2}^2+{\omega_3}^2+{\omega_4}^2)\cdot\sin(\theta)} &\\
	\eq{m\cdot\ddot{y}_I}{-k_{th}\cdot({\omega_1}^2+{\omega_2}^2+{\omega_3}^2+{\omega_4}^2)\cdot(-\sin(\phi))\cdot\cos(\theta)} &\\
	\eq{m\cdot\ddot{z}_I}{F_g-k_{th}\cdot({\omega_1}^2+{\omega_2}^2+{\omega_3}^2+{\omega_4}^2)\cdot\cos(\phi)\cdot\cos(\theta)}
	\label{eq:AccelerationEqInertialVelocities}
\end{flalign}
%
\small
\begin{flalign}
	\si{R^{-1}_{\phi, \theta, \psi}} &=
	\begin{bmatrix}
		\ \si{c(\theta) \cdot c(\psi)}                & \si{-c(\theta) \cdot s(\psi)}  & \si{-c(\phi) \cdot s(\theta) \cdot c(\psi) + s(\phi) \cdot s(\psi)}  \ \ \ \\ 
		\ \si{s(\phi) \cdot s(\theta) \cdot s(\psi) + c(\phi) \cdot s(\psi)}  	  & \si{-s(\phi) \cdot s(\theta) \cdot s(\psi) + c(\phi) \cdot c(\psi)} 		& \si{c(\phi) \cdot s(\theta) \cdot s(\psi) + s(\phi) \cdot c(\psi)}                \ \ \ \\ 
		\ \si{s(\theta)}      	  & \si{-s(\phi) \cdot c(\theta)}    		& \si{c(\phi) \cdot c(\theta)}                 \ \ \ 
	\end{bmatrix} 	\label{invRotMatrix}
\end{flalign}
\normalsize
