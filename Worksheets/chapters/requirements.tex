\chapter{Functional Requirements}
\label{ch:functionalRequirements}
Based on the prototype description in  \autoref{sec:PrototypeDescription}, functional requirements can be established for the prototype.

%be possible to send a new position reference for the prototype wirelessly.
\begin{enumerate}[label=\textbf{\arabic*})]

%\begin{enumerate}[label=\textbf{\arabic*})]
\item \textbf{The quadcopter should be able to receive its own position and attitude from the Vicon system each control loop, through a computer at the ground station.}
\begin{itemize}
\item[] %It is necessary for the quadcopter to receive it own position and attitude. 
This is essential for controlling and navigating the quadcopter. Additionally, it should be possible for the quadcopter to receive the information through a wireless channel and discard corrupted packets.  %, and thereby be able to fly freely.
\end{itemize}

%\item \textbf{The prototype should be able to disregard incorrect packets received from the ground computer.}
%\begin{itemize}
%\item[] To ensure that invalid data is not utilized, it shall be possible for the processor on the quadcopter to detect corrupt packets received from the ground station.
%\end{itemize}

%\item \textbf{It shall be able to wireless receive commands sent from the computer at the ground station.}
%\begin{itemize}
%\item[] It must be possible to demand that the quadcopter either takes off or lands. It must also be possible to send a new position for the quadcopter to fly to.  
%\end{itemize}

\item \textbf{It shall be possible to control the quadcopter's attitude.}
\begin{itemize}
\item[] To be able to stabilize the quadcopter, it is essential that it is possible to control the attitude. This is essential if the quadcopter is changing position in either the x, y, or z axes or when it needs to maintain its position. 
\end{itemize}

\item \textbf{It shall be possible to control the position in the z axis.}
\begin{itemize}
\item[] It should be possible for the quadcopter to keep itself in a fixed z position and be able to move in the z axis. 
\end{itemize}

\item \textbf{It shall be possible to control the quadcopter's position in the x and y axes}
\begin{itemize}
\item[] It should be possible for the quadcopter to keep itself in a fixed x,y position and be able to move in these two axes. 
\end{itemize}

\end{enumerate}























%\item \textbf{It must be able to take off.\fxnote{Whis we can not test it}}
%\begin{itemize}
%\item[]The quadcopter must be able to take off without flipping or crashing in order to make the quadcopter fly and fulfil the other requirements.
%\end{itemize}
%\newpage
%\item \textbf{It must be able to land.\fxnote{Whis we can not test it}}
%\begin{itemize}
%\item[]The quadcopter must be able to land without flipping or crashing. It must be able to touch down without overshoot, as this may damage the quadcopter if the impact with the ground is too great.
%\end{itemize}


%\item \textbf{It must be able to land automatically, when the battery level is below 10 V.\fxnote{Whis we can not test it}}
%\begin{itemize}
%\item[] This is necessary in order to avoid crashing due to engine failure\fxnote{engine failure?}. The quadcopter will thereby only operate when the battery delivers the required power for the four motors. This is to avoid damaging the quadcopter and the battery.
%\end{itemize}


%\item \textbf{It shall be possible to change the position of the quadcopter in the y axis.}
%\end{itemize}

%\item \textbf{It shall be possible to change the position of the quadcopter in the y axis.}
%\begin{itemize}
%\item[] .
%\end{itemize}

%\item \textbf{It shall be possible to change the position of the %quadcopter in the y axis.}
%\end{itemize}

%\item \textbf{It shall be possible for the quadcopter to land.}
%\end{itemize}

%\item \textbf{The quadcopter should be able to land if the battery voltage is below 10 volts.}
%\begin{itemize}
%\item[] This is to ensure that it is possible for the quadcopter to land safely on ground before the battery is below 10 volts. The quadcopter will thereby only operate when the battery can deliver the required power for the four motors. This is to avoid damaging the quadcopter.
%\end{itemize}

%\item \textbf{It shall be possible for the quadcopter to take of.}
%\end{itemize}

%The purpose of this chapter is to consider requirements for the system, that will function as guideline when the control system is to be designed at a later point.
%
%The quadcopter has to be able to fly and hover stabalily. It must be able to withstand a disturbance of a certain size.\fxfatal{This is to be settled by choice} 
%Requirements for the system's steprespons  is to be set, as this will be the dynamic behaviour of the system. 
%
%The quadcopter must be able to obtain stability after it has been given an impulse, as this will simulate a disturbance.
%
%The user gives a position input at the ground station, that is send to the quadcopter. The controllers are run on board. To simplify matters as the three axis are coupled, the controllers shall get the quadcopter to the requested position by first moving it in one axes, then the other and lastly the third. This is not an efficient movement, but due to the time constraint of the project, it is not considered reasonable to obtain a robust control system that will be able to change all three axis while maintaining stable. \fxfatal{you will probably have last part removed, i just wrote it quickly - no harm done ;)}
%
%To be written:
%
%Time domain\\
%What will overshoot mean for the system physically ? \\
%What will a steady state error mean? \\
%What will a slow rise time mean? Slow system? Do we want a really fast system. Fast and stable is a compromise - how should that compromise be levelled in our project? 
%
%Frequency domain\\
%Phase margin and gain margin - how robust is the controller. How far is the system from instability? 