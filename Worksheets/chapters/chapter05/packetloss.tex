\section{Package Loss}
The loss of packages in the system causes the controller to use old data in the loop. This effect is also included in the simulation by utilizing TrueTime. In the system, package loss is caused by the scheduling of the tasks in the microcontroller since information can not be processed when the controllers are running. Package loss across the wireless channel does not occur due to the low transmission frequency used. 

In the network simulation tool, package loss is simulated as a constant probability of loosing packages. To get this probability, tests are conducted in which 10000 packages are sent. The amount of packages received and processed in the microcontroller and the number of control loops executed gives an estimate of the probability of losing packages. The data obtained with the tests is shown in more detail in \fxnote{APPENDIX WITH THE TEST}. The used value for the package loss probability is set to be 0 as in all tests, there were more packages received than control loops executed. This implies that the most recent data is available for the control in every loop. 