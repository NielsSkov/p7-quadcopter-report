\chapter{Network} \label{ch:Network}

The control of a quadcopter implies, in most cases, the use of a wireless network. In the prototype at hand, the information transmitted to the microcontroller is the attitude, translational position and position reference of translational controllers and the translational velocity of the quadcopter.

The control of such a system trough a network requires the design of a communication protocol. The use of the XBEE modules makes the design easier as only the transport layer must be taken into account. As typical for real time systems, this will be a connectionless protocol to ensure the data transmission is as fast as possible.

The effects of the network in the control system are analyzed using the TrueTime toolbox. The two issues considered are the delay and the package loss as these disturb the control system the most.
%The Vicon system has a sampling rate of 100 Hz. A connectionless protocol can thereby be utilized, as it is sufficient to transmit the latest data to the micro controller instead of asking for a retransmission, e.g. if a package is lost or corrupt. Furthermore a retransmission will cause a delay in the system, this is not desired as this real time system is sensitive to delay.