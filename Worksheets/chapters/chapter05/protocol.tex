\section{Communication Protocol}
The main information necessary to transmit from the computer to the micro controller, is the position, attitude and translational velocity of the quadcopter. A created transport layer protocol is utilized to ensure transmitted packages can be detected and that erroneous packages is not utilized by the micro controller.

The Vicon system has a sampling rate of 100 Hz. A connectionless protocol can thereby be utilized, as it is sufficient to transmit the latest data to the micro controller instead of asking for a retransmission, e.g. if a package is lost or corrupt. Furthermore a retransmission will cause a delay in the system, this is not desired as this real time system is sensitive to delay.

\subsection{Package Structure}
The communication link is only between two nodes, i.e. the computer and the micro controller. The protocol thus therefore not need either a source port or destination addresses, as typically seen in a connectionless UDP protocol. Furthermore, it is decided not to change the length of the package relative to the gathered data.\\
If a package is lost, the loss is not detected. However, this could be done by implementing a sequence number in the header of the package, but since the protocol is decided to be a connectionless protocol, it is deemed unnecessary. As the length is constant only a start sequence and not an end sequence is necessary. These considerations simplifies the protocol header and tail. It is decided that for the prototype a checksum is enough for detecting corrupted packages. An overall structure of a generated package is illustrated in \autoref{tab:Packagestructure}.

\begin{table}[H]\centering
\begin{tabular}{|>{\centering\arraybackslash}m{3cm}|>{\centering\arraybackslash}m{2cm}|>{\centering\arraybackslash}m{2cm}|}
\hline
Start Sequence & Data & Checksum \\
\hline
\end{tabular}
\caption{The structure of a package}
\label{tab:Packagestructure}
\end{table}

\subsubsection{Start Sequence}
A start sequence is necessary for the micro controller to be able to differentiate between packages and noise. As the micro controller receives packages from the computer, a queue of packages stored on the micro controller together with some noise is generated. The start sequence enables the system to scan the information in the queue for the specific start sequence. If the start sequence is found the micro controller can extract the package, as the length is fixed. It has not been deemed necessary to have an end byte, as the package length is fixed and the checksum ensures that an incorrect packages is not utilized. Thus it is deemed redundant to check if the end of the package is the desired end.

The start sequence is set to three bytes. The chosen length of the start sequence is explain later in checksum. To ensure the unlikeliness of finding a start sequence in the middle of a package all the bits of the start sequence is set to ones. As it is highly unlikely for the Vicon system to register two of the transmitted variables, which are place beside each other in the package, to be all ones at the same time. The header (the start sequence) can be seen included in \autoref{tab:Packagestructure}.

%The queue of packages could be due to a slow package handler, as the sampling rate is set to low \fxnote{true? maybe som actual values}.
\subsubsection{Data}
The variable which are transmitted from the computer to the micro controller, are the attitude (roll, pitch and yaw), the position (x, y and z) and the translational velocity ($\dot{x}$, $\dot{y}$ and $\dot{z}$).

The size of the roll, pitch and yaw are measured in radians and is set to be 9 bits each. The first bit of the nine is set up as the sign bit. As radians operates in decimal numbers, the number is multiplied by 100 to not remove essential data. Yielding a maximum value of 314, rounded down. This value is only measured by the Vicon system if the quadcopter does a 180 degree turn. If this occurs in the roll and pitch, it is not possible for the prototype to recover from being turned upside down. Thus a byte is enough for containing the necessary data, as a 90 degree turn yield a value of 157. However, if the quadcopter rotates 180 degrees in yaw, 9 bites are needed to contain the variable. Nevertheless, in this prototype the controller is not implemented for a 180 degree rotated yaw, this is discuss further in \autoref{chap:Control} \fxnote{discuss this in control}. One byte is therefore sufficient for the yaw. When the package is received the micro controller casts the 9 bits to a float and divides by a 100.

The position coordinates (x, y and z) are measured in millimetre by the vicon system and are set in the protocol to have a size of 10 bits each. The first bit is utilized as a sign bit. It is deemed unnecessary to measure the distance in millimetre. By changing the variable to centimetre, 9 bits is sufficient, as values between 512 and minus 512 yields a maximum distance around 5 meters. Thus, making it sufficient for this prototype, as the quadcopter is operated in the Vicon room, see \autoref{cha:Systemdescription} \fxnote{ensure the dimensions of the vicon room is written in the system description}.

The translational velocity ($\dot{x}$, $\dot{y}$ and $\dot{z}$) 




 % It should be noted that both the computer and the micro controller are little endians. The most significant bit is therefore the sign bit in both systems. 


\begin{table}[H]
\centering
\begin{tabular}{llclllllllllllllll}
\hline
\multicolumn{2}{|c|}{Offset}                          & \multicolumn{8}{c|}{Byte 1}                                                                                                                                                                           & \multicolumn{8}{c|}{Byte 2}                                                                                                                                                                                 \\ \hline
\multicolumn{1}{|l|}{Byte} & \multicolumn{1}{l|}{Bit} & \multicolumn{1}{c|}{0} & \multicolumn{1}{c|}{1} & \multicolumn{1}{c|}{2} & \multicolumn{1}{c|}{3} & \multicolumn{1}{c|}{4} & \multicolumn{1}{c|}{5} & \multicolumn{1}{c|}{6} & \multicolumn{1}{c|}{7} & \multicolumn{1}{c|}{8} & \multicolumn{1}{l|}{9} & \multicolumn{1}{l|}{10} & \multicolumn{1}{l|}{11} & \multicolumn{1}{l|}{12} & \multicolumn{1}{l|}{13} & \multicolumn{1}{l|}{14} & \multicolumn{1}{l|}{15} \\ \hline
\multicolumn{1}{|l|}{0}    & \multicolumn{1}{l|}{0}   & \multicolumn{16}{c|}{Header}                                                                                                                                                                                                                                                                                                                                                                                        \\ \hline
\multicolumn{1}{|l|}{2}    & \multicolumn{1}{l|}{16}  & \multicolumn{8}{l|}{}                                                                                                                                                                                 & \multicolumn{8}{c|}{$\phi$}                                                                                                                                                                                 \\ \hline
\multicolumn{1}{|l|}{4}    & \multicolumn{1}{l|}{32}  & \multicolumn{1}{c|}{}  & \multicolumn{9}{c|}{$\theta$}                                                                                                                                                                                                  & \multicolumn{6}{c|}{$\psi$}                                                                                                                               \\ \hline
\multicolumn{1}{|l|}{6}    & \multicolumn{1}{l|}{48}  & \multicolumn{3}{c|}{}                                                    & \multicolumn{10}{c|}{$x$}                                                                                                                                                                                                                                  & \multicolumn{3}{l|}{}                                                       \\ \hline
\multicolumn{1}{|l|}{8}    & \multicolumn{1}{l|}{64}  & \multicolumn{7}{c|}{$y$}                                                                                                                                                     & \multicolumn{9}{c|}{$z$}                                                                                                                                                                                                             \\ \hline
\multicolumn{1}{|l|}{10}   & \multicolumn{1}{l|}{80}  & \multicolumn{1}{c|}{}  & \multicolumn{10}{c|}{$x_{ref}$}                                                                                                                                                                                                                          & \multicolumn{5}{c|}{$y_{ref}$}                                                                                                  \\ \hline
\multicolumn{1}{|l|}{12}   & \multicolumn{1}{l|}{96}  & \multicolumn{5}{c|}{}                                                                                                      & \multicolumn{10}{c|}{$z_{ref}$}                                                                                                                                                                                                                              & \multicolumn{1}{l|}{}   \\ \hline
\multicolumn{1}{|l|}{14}   & \multicolumn{1}{l|}{112} & \multicolumn{10}{c|}{$\dot{x}$}                                                                                                                                                                                                                         & \multicolumn{6}{c|}{$\dot{y}$}                                                                                                                            \\ \hline
\multicolumn{1}{|l|}{16}   & \multicolumn{1}{l|}{128} & \multicolumn{5}{l|}{}                                                                                                      & \multicolumn{11}{c|}{$\dot{z}$}                                                                                                                                                                                                                                                        \\ \hline
\multicolumn{1}{|l|}{18}   & \multicolumn{1}{l|}{144} & \multicolumn{16}{c|}{Checksum}                                                                                                                                                                                                                                                                                                                                                                                      \\ \hline
\multicolumn{1}{|l|}{20}   & \multicolumn{1}{l|}{160} & \multicolumn{8}{c|}{}                                                                                                                                                                                 & \multicolumn{1}{l|}{}  & \multicolumn{1}{l|}{}  & \multicolumn{1}{l|}{}   & \multicolumn{1}{l|}{}   & \multicolumn{1}{l|}{}   & \multicolumn{1}{l|}{}   & \multicolumn{1}{l|}{}   & \multicolumn{1}{l|}{}   \\ \hline
                           &                          & \multicolumn{1}{l}{}   &                        &                        &                        &                        &                        &                        &                        &                        &                        &                         &                         &                         &                         &                         &                         \\
                           &                          & \multicolumn{1}{l}{}   &                        &                        &                        &                        &                        &                        &                        &                        &                        &                         &                         &                         &                         &                         &                         \\
                           &                          & \multicolumn{1}{l}{}   &                        &                        &                        &                        &                        &                        &                        &                        &                        &                         &                         &                         &                         &                         &                        
\end{tabular}
\caption{My caption}
\label{tab:Packagestructure}
\end{table}






\subsubsection{Checksum}







