\chapter{Acceptance test}

It must be tested if the designed networked control system complies with the requirements mentioned in \autoref{ch:technicalRequirements}. This is done in the following. %Before the test can be performed pass/fail criteria and procedures needs to be formulated. 

There are three assessment degrees on how well the developed prototype satisfies the requirements. The degrees are \ding{51} for a pass, (\ding{51}) for a partial pass and \ding{55} for a fail. 

\subsection*{Requirements Overview}
For repetition, the requirements that need to be met are:
\begin{enumerate}[label=\textbf{\arabic*})]
\item {The quadcopter should be able to receive its own position and attitude from the Vicon system each control loop, through a computer at the ground station.}
\item {The prototype should be able to disregard incorrect packets received from the computer.}
\item {The quadcopter should be able to land if the battery voltage is below 10 volts.}
\item {It shall be possible to control the quadcopters attitude.}
\item {It shall be possible to make the quadcopter hover and control its height while hovering.}
\item {It shall be possible to change the position of the quadcopter in the $x_{\mathrm{I}}$, $y_{\mathrm{I}}$ and $z_{\mathrm{I}}$ axis.}
\end{enumerate}

\newpage
\subsection*{Equipment}
To perform the tests, the following test equipment is used:
\begin{table}[H] \centering
\resizebox{\textwidth}{!}{
\begin{tabular}{|l|l|l|} 
\hline 
\textbf{Equipment} &  \textbf{AAU-no.} & \textbf{Type/notes} \\ 
\hline 
2x USRP & 100784 \& 100785 & N210\\ 
\hline 
0.5 m SMA cable & - &- \\ 
\hline 
Attenuators & - &  1x20 dB and 2x10 dB \\ 
\hline 
MIMO cable & - &-\\\hline 
Ethernet cable & - & Must support gigabit ethernet \\ 
\hline 
LabVIEW compatible with USRP N210 & - & 2015\\ 
\hline 
A computer with gigabit ethernet &- & MSI GP60 2PE Leopard Pro\\ \hline 
2x power supplies for USRPs &- & 6V DC \\
\hline 
Bi-directional coupler & Label: 1469-01 & ZGDC35-93HP+\\ \hline 
\end{tabular} 
}
\caption{Test equipment used for the acceptance test.}
\label{tab:test_equipment}
\end{table}

\subsection*{General Setup}\label{sec:General_setup}
For all tests, the general setup of the systems is explained in the following: 


\section{Utilize new sensor data in each control loop}
\textbf{Requirement:}
\textit{The quadcopter should be able to receive its own position and attitude wirelessly from the Vicon system each control loop, through a computer at the ground station.}

\textbf{Pass/fail criteria:}
	\begin{description}
	\item[ \ding{51} ] One new decoded packet is utilized in each control loop.
	\item[(\ding{51})] 99 percent of the control loops runs with the recent received data from a new decoded packet. Indicating that 1 percent of the time a control loop runs with old data.\fxnote{should we come up with numbers?}
	\item[ \ding{55} \phantom{)}] More than 1 percent of the control loops runs with old data.
	\end{description}

		
\textbf{Procedure:}\\

\begin{enumerate}
	\item Enable the computer, utilizing Matlab, to transmit a 1000 packets to the micro processor.
	\item The data in the first transmitted packet consist of zeros, hereafter the data in each packet is incremented. The last packet will then contain the number 999. 
	\item In every control loop the data which is utilized should be compared to the data utilized in the loop before. If the data is not identical a counter should be incremented. 
	\item The comparison should only take place if the communication task register a packet containing data which is less or equal to than 999.
	\item The first packet consisting of zeros should not be compared to an old packets (as there is none) but is instead compared to the number 1001. 
	\item The counter, which is incremented each time the data utilized in the control loop is not identical to the previous, should be transmitted back to the computer when the computer is done transmitting.
	\item A counter incremented each control loop while packets are received should also be transmitted back.
\end{enumerate} 

\textbf{Results:}

\newpage

\section{Disregard incorrect packages received from the computer}
\textbf{Requirement:}
\textit{The prototype should be able to disregard incorrect packets received from the computer.}

\textbf{Pass/fail criteria:}
	\begin{description}
	\item[ \ding{51} ] The receiver detects all transmitted packages at the beginning of the package, which has an SNR of max 2.8 dB and the receiver does not trigger when no package has been transmitted (false positives).
	\item[(\ding{51})]The receiver is able to detect transmitted packages at the beginning of the package, which has an SNR of max 2.8 dB, but false positives may occur.
	\item[ \ding{55} \phantom{)}]The receiver is not able to detect any transmitted packages.
	\end{description}
	
\textbf{Procedure:}\\
\begin{enumerate}
	\item ..
	\item ..
	\item ..
\end{enumerate} 


\textbf{Results:}


\newpage

\section{Land if battery is below 10 volts}
\textbf{Requirement:}
\textit{The quadcopter should be able to land if the battery voltage is below 10 volts.}

\textbf{Pass/fail criteria:}
	\begin{description}
	\item[ \ding{51} ] The receiver detects all transmitted packages at the beginning of the package, which has an SNR of max 2.8 dB and the receiver does not trigger when no package has been transmitted (false positives).
	\item[(\ding{51})]The receiver is able to detect transmitted packages at the beginning of the package, which has an SNR of max 2.8 dB, but false positives may occur.
	\item[ \ding{55} \phantom{)}]The receiver is not able to detect any transmitted packages.
	\end{description}

		
\textbf{Procedure:}\\


\begin{enumerate}
	\item ..
	\item ..
	\item ..
\end{enumerate} 


\textbf{Results:}


\newpage

\section{Control the quadcopters attitude}
\textbf{Requirement:}
\textit{It shall be possible to control the quadcopters attitude.}

\textbf{Pass/fail criteria:}
	\begin{description}
	\item[\ding{51}] The attitude of the quadcopter is stabilized around 0 rad.
	\item[(\ding{51})] The attitude of the quadcopter is stabilized around 0 with oscillations and is not able to handle disturbances.
	\item[\ding{55} \phantom{)}] The control system does not stabilize the attitude of the quadcopter or it is unstable.
	\end{description}
\textbf{Procedure:}\\
\begin{enumerate}
	\item Hang the quadcopter from the ceiling with a string and program the attitude controller on the microcontroller.
	\item Start the transmission of attitude data from the Vicon system to the quadcopter.
	\item Record the data given by the Vicon system during the test.
\end{enumerate} 
\textbf{Results:}


\newpage

\section{Control the height in which the quadcopter is flying}
\textbf{Requirement:}
\textit{It shall be possible to make the quadcopter hover and control the height of the hovering.}

\textbf{Pass/fail criteria:}
	\begin{description}
	\item[ \ding{51} ] The receiver detects all transmitted packages at the beginning of the package, which has an SNR of max 2.8 dB and the receiver does not trigger when no package has been transmitted (false positives).
	\item[(\ding{51})]The receiver is able to detect transmitted packages at the beginning of the package, which has an SNR of max 2.8 dB, but false positives may occur.
	\item[ \ding{55} \phantom{)}]The receiver is not able to detect any transmitted packages.
	\end{description}

		
\textbf{Procedure:}\\


\begin{enumerate}
	\item ..
	\item ..
	\item ..
\end{enumerate} 


\textbf{Results:}


\newpage

\section{Control the position in the x, y and z axis}
\textbf{Requirement:}
\textit{It shall be possible to change the position of the quadcopter in the x, y and z axis.}

\textbf{Pass/fail criteria:}
	\begin{description}
	\item[ \ding{51} ] The receiver detects all transmitted packages at the beginning of the package, which has an SNR of max 2.8 dB and the receiver does not trigger when no package has been transmitted (false positives).
	\item[(\ding{51})]The receiver is able to detect transmitted packages at the beginning of the package, which has an SNR of max 2.8 dB, but false positives may occur.
	\item[ \ding{55} \phantom{)}]The receiver is not able to detect any transmitted packages.
	\end{description}

		
\textbf{Procedure:}\\


\begin{enumerate}
	\item ..
	\item ..
	\item ..
\end{enumerate} 


\textbf{Results:}

\newpage
\section{Summary of Results}
The results of the tests are summed up in \autoref{tab:acceptance_test_results}.
\begin{table}[H] \centering
\begin{tabular}{|c|p{11cm}|c|}
\hline 
\textbf{Req. nr.} & \textbf{Requirement} & \textbf{Result} \\ 
\hline 
1 & The quadcopter should be able to receive its own position and attitude wirelessly from the Vicon system, through a computer. & \ding{51}\\ 
\hline
2 & The prototype should be able to disregard incorrect packets received from the computer. & \ding{51} \\ 
\hline 
3 & The quadcopter should be able to land if the battery voltage is below 10 volts. & \ding{51} \\ 
\hline 
4 & It shall be possible to control the quadcopters attitude. & \ding{51} \\ 
\hline 
5 & It shall be possible to make the quadcopter hover and control the height of the hovering. & \ding{51}  \\ 
\hline  
6 & It shall be possible to change the position of the quadcopter in the x, y and z axis. & \ding{51} \\ 
\hline  
\end{tabular} 
\caption{Summary of acceptance test results.}
\label{tab:acceptance_test_results}
\end{table}

As can be seen, the prototype fulfils 8/8 of the set requirements. 


