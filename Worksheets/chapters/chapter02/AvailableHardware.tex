\section{Available Hardware}

\subsection{Motor and Propeller}

\subsection{Platform}

\subsection{Battery}

The battery, which is available for the prototype, is a Zippy Flightmax battery. It weights 141 gram, has a capacity of 1500 mAh, a voltage of 11.1 volts and a discharge current of 20 amperes. 

\subsection{Vicon System}

The Vicon system is a powerful tool that provides real-time position and orientation data captured with infrared cameras.This information can be used to track objects inside the room the system is built in. An example is found in Aalborg University as seen in \figref{ViconRoom}. 
%\begin{figure}[H]
%	\centering
%	\includegraphics[scale=0.5]{figures/ViconRoom}
%	\caption{Aalborg University's }
%	\label{ViconTracker}
%\end{figure}
For using this system, markers are attached to the object that is to be tracked. The Vicon system streams the position of the markers and the position and orientation of the object at 100 Hz for a computer to read them.

In \figref{ViconTracker} the user interface with the Vicon system is shown. It is called Vicon Tracker and it allows the creation of objects from the markers present in the room. It allows also to change the center of gravity of the created objects and to rotate the reference frames to a desired position.
%\begin{figure}[H]
%	\centering
%	\includegraphics[scale=0.5]{figures/ViconTraker}
%	\caption{User interface of the Vicon System, the Vicon Tracker. An object has been created and its center of gravity is being modified.}
%	\label{ViconTracker}
%\end{figure}
