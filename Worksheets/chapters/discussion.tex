\chapter{Discussion}
The main limiting effects in the control design are the delays and the maximum frequency in which packets can be transmitted to the microcontroller. These two factors make the attitude control loop's bandwidth low. The translational position and velocity controllers are also affected as the bandwidths of these need to be lower than the bandwidth of the attitude loop. This issue can be solved by implementing on board sensors to obtain the attitude of the quadcopter. In this way, the control loop for the attitude controller could run faster and the delays would be smaller, thus allowing a faster response of the system.

From the tests performed, it has been observed that the translational controllers do not perform as expected in the real prototype. The most probable cause for the problem is an implementation error or a hardware issue that is yet to be identified. It has been showed that the design approach resulted in working translational controllers as the simulations yield successful results. 

The way of modeling the system is deemed suitable for the purpose of controlling the quadcopter. A more detailed model could yield a more reliable simulation when testing the controllers.

The used microcontroller runs at 16M IPS. This is deemed sufficient but it is possible to find microcontrollers with better performance. A higher execution rate or a device with parallel capabilities such as a FPGA could yield a better performance of the system. 

As mentioned in \autoref{subsec:ESC}, the motor controllers do not take into account the voltage level of the battery when running. This constitutes a disturbance for the controllers as the rotational speed demanded is not attained by the motors. This issue does not stop the controllers from working, but the performance will be less consistent in sustained flight. It could be solved by measuring and accounting for the battery voltage when setting the duty reference for the motor controllers. Another solution could be to use different motor controllers with continuous battery level compensation.

%implementing a battery voltage measurement in the microcontroller code that used this measurement to calculate the required duty cycle to the ESCs so the speed of the motors is the same as that required by the controller.


%\begin{itemize}


%The Vicon data tracking system samples with a maximum effective frequency of 100 Hz, which limits the control loops' bandwidth. This results in a rather slow position control, as it is the outer most loop.\\
%By implementing attitude on board sensors, the network delay introduced by using the Vicon system, the bandwidth of the control loops can be increased and thereby obtaining more agile control.\\
%The bandwidth of the controllers can also be increased by implementing a faster microprocessor, that will minimize the processing time.

%\item Translational controllers does not work in reality but only simulation, why?:\\
%It may be due to an implementation error, that has yet to be found and eliminated.\\
%An error may be, that the controllers are feeded with amounts of old data that makes it impossible for the controllers to stabalize the quadcopter due to too large delays in the feeded data.\\
%Another suggestion is too much packet loss. \fxnote{we does not know the amount of packet loss yet - tbd}

%\item scheduling, faster than the cycle of the control algorithm 

%\item ground effect (turbulens), issue experinced

%\item battery level issue: implement function
  

%\end{itemize}