\section{State Space Representation for the Angular Model Equations}
\begin{minipage}{0.32\linewidth}
	\begin{flalign}
		\vec{x(t)} = 
		\begin{bmatrix}
			\phi \\
			\theta \\ 
			\psi \\
			\dot{\phi} \\
			\dot{\theta} \\
			\dot{\psi} \\
		\end{bmatrix}	\nonumber
		\label{xVector}
	\end{flalign}  
\end{minipage}\hfill
%\hspace{0.03\linewidth}
\begin{minipage}{0.32\linewidth}
	\begin{flalign}
		\vec{y(t)} = 
		\begin{bmatrix}
			\phi \\
			\theta \\ 
			\psi \\
		\end{bmatrix}	\nonumber
		\label{yVector}
	\end{flalign}
\end{minipage}\hfill
%\hspace{0.03\linewidth}
\begin{minipage}{0.32\linewidth}
	\begin{flalign}
		\vec{u(t)}= 
		\begin{bmatrix}
			\omega_1 \\
			\omega_2 \\
			\omega_3 \\
			\omega_4 \\
		\end{bmatrix}
		\label{uVector}
	\end{flalign} \nonumber
\end{minipage}\hfill

The relationship between them is  given in the form of first-order differential equations:
\begin{flalign}
	\eq{\vec{\dot{x}}(t)}{f(\vec{x}(t),\vec{u}(t))}
	\label{xDotDiffEq} 
\end{flalign}
\begin{flalign}
	\eq{\vec{y}(t)}{g(\vec{x}(t),\vec{u}(t))} 
	\label{yDiffEq} 
\end{flalign}
%
If the system can be described with linear equations, the previous representation can be written in matrix form, giving \eqref{xDotLinear} and \eqref{yLinear}.
%
\begin{flalign}
	\eq{\vec{\dot{x}}(t)}{\vec{A} \cdot \vec{x}(t) + \vec{B} \cdot \vec{u}(t)}
	\label{xDotLinear} 
\end{flalign}
\begin{flalign}
	\eq{\vec{y}(t)}{\vec{C} \cdot \vec{x}(t) + \vec{D} \cdot \vec{u}(t)}
	\label{yLinear} 
\end{flalign}
%
\hspace{6mm} Where:\\
\begin{tabular}{ p{1cm} l l l}
	& \si{\vec{A}=\frac{\partial}{\partial \vec{x}} \ f(\vec{x_o},\vec{u_o})}			& is the \si{6x6}  state matrix     \\                       
	& \si{\vec{B}=\frac{\partial}{\partial \vec{u}} \ f(\vec{x_o},\vec{u_o})}			& is the \si{6x4}  input matrix       \\ 
	& \si{\vec{C}=\frac{\partial}{\partial \vec{x}} \ g(\vec{x_o},\vec{u_o})}			& is the \si{3x6}  output matrix      \\ 
	& \si{\vec{D}=\frac{\partial}{\partial \vec{u}} \ g(\vec{x_o},\vec{u_o})}			& is the \si{3x4}  feedforward matrix \\ 
\end{tabular} 
\\

This state space description can be seen also in the form of a block diagram like the one in \figref{SSBlocks}.
%
\begin{figure}[H]
	\begin{tikzpicture}[ auto,
                       thick,                         %<--setting line style
                       node distance=1.5cm,             %<--setting default node distance
                       scale=1,                     %<--|these two scale the whole thing
                       every node/.style={scale=1}, %<  |(always change both)
                       >=triangle 45 ]                %<--sets the arrowtype
    
    \draw%-----------------------------------------------------------------------------------------
    	%Drawing Input/Output:
    	node[shape=coordinate][](input1) at (0,0){}
    	node[shape=coordinate][](output1) at (9.5,0){}
     	%Drawing the Equation Blocks:   	
      	node(A) at (4.5,-1.5) [block] {A} 
     	node(B) at (1.5,0) [block] {B}
     	node(C) at (6.5,0) [block] {C}
      	node(D) at (4.5,1.5) [block] {D}  
	    node(int) at (4.5,0) [block] {\si{\int}}  
    	%Drawing the Sumation Blocks:	    	 	
    	node(sum1) [sum, right of = B] {\si{\sum}}
    	node(sum2) [sum, right of = C] {\si{\sum}}
    	%Drawing the Feedback/Feedforward Nodes:    	
    	node[shape=coordinate][](FeedforwardNode) at (0.75,0){}
    	node[shape=coordinate][](FeedbackNode) at (5.5,0){}  	
    	     
    ;%---------------------------------------------------------------------------------------------
   
    %Joining the Blocks
  	\draw[->](input1) -- node {u}(B);
  	\draw[->](B) -- node {}(sum1);
  	\draw[->](sum1) -- node {\si{\dot x}}(int);  	
  	\draw[->](int) -- node {x}(C);
  	\draw[->](C) -- node {}(sum2);  	
  	\draw[->](sum2) -- node {y}(output1);
  	
  	\draw[->](FeedforwardNode) |- node{} (D);
  	\draw[->](D) -| node{} (sum2);

  	\draw[-] (FeedbackNode) |- (A);
  	\draw[->] (A)   -| (sum1);

    %Drawing node(s) with \textbullet
    \draw%--------------------------------------------------------------
      node at (input1)  [shift={(-0.08, -0.02 )}] {\large \textbullet}
    	% node at (output1) [shift={( 0.008, -0.02 )}] {\textbullet}
    ;%------------------------------------------------------------------
  \end{tikzpicture}
	\centering
	\caption{Block diagram of the state space representation of the system.}
	\label{SSBlocks}
\end{figure}\vspace{-18pt}
%
The specific matrices for the description of the angular behavior can be obtained from the linearized equations of the system (\eqref{eqAngleLin}), given the final system description as \eqref{xDotSS} and \eqref{ySS}.

\begin{flalign}   \label{xDotSS}
	\dot{x}(t) &=
	\begin{bmatrix}
		\ 0 & 0 & 0 & 1 & 0 & 0     \ \ \ \\ 
		\ 0 & 0 & 0 & 0 & 1 & 0     \ \ \ \\ 
		\ 0 & 0 & 0 & 0 & 0 & 1     \ \ \ \\
		\ 0 & 0 & 0 & 0 & 0 & 0     \ \ \ \\ 
		\ 0 & 0 & 0 & 0 & 0 & 0     \ \ \ \\ 
		\ 0 & 0 & 0 & 0 & 0 & 0     \ \ \ 		
	\end{bmatrix}
	x(t) +
	\begin{bmatrix}
		\ 0 & 0 & 0 & 0      \ \ \ \\ 
		\ 0 & 0 & 0 & 0      \ \ \ \\ 
		\ 0 & 0 & 0 & 0      \ \ \ \\
		\ 0 & \si{-\frac{2 \cdot k_{th} \cdot L \cdot \overline{\omega}_2}{J_x}} & 0 & \si{\frac{2 \cdot k_{th} \cdot L \cdot \overline{\omega}_4}{J_x}}      \ \ \ \\ 
		\ \si{\frac{2 \cdot k_{th} \cdot L \cdot \overline{\omega}_1}{J_y}} & 0 & \si{-\frac{2 \cdot k_{th} \cdot L \cdot \overline{\omega}_3}{J_y}} & 0      \ \ \ \\ 
		\ \frac{2 \cdot k_d \cdot {\overline{\omega}_1}}{J_z} & - \frac{2 \cdot k_d \cdot {\overline{\omega}_2}}{J_z} & \frac{2 \cdot k_d \cdot {\overline{\omega}_3}}{J_z} & - \frac{2 \cdot k_d \cdot {\overline{\omega}_4}}{J_z}      \ \ \ 		
	\end{bmatrix}
	u(t)
\end{flalign}
\begin{flalign} \label{ySS}
	\si{y(t)} &=	 
	\begin{bmatrix}
		\ 1 & 0 & 0 & 0 & 0 & 0     \ \ \ \\ 
		\ 0 & 1 & 0 & 0 & 0 & 0     \ \ \ \\ 
		\ 0 & 0 & 1 & 0 & 0 & 0     \ \ \ 		
	\end{bmatrix}
	x(t)
\end{flalign}

The dynamics of the system can be analysed looking at the matrices, since


