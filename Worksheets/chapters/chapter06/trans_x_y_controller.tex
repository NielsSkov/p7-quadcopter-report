\subsection{Controllers for x and y Axes}
The design of the controllers for the translational movement in $x_I$ and $y_I$ directions is done through the use of the root locus design method.

The linear equations for the translational model relate this movements with the roll and the pitch that the quadcopter has at each moment.
%
\begin{flalign}
    m\ \Delta\ddot{x}_I &= -k_{th}\ ({\overline{\omega}_1}^2+{\overline{\omega}_2}^2+{\overline{\omega}_3}^2+{\overline{\omega}_4}^2)\ \Delta\theta \label{eq:model_x_transl} \\
    m\ \Delta\ddot{y}_I &=  k_{th}\ ({\overline{\omega}_1}^2+{\overline{\omega}_2}^2+{\overline{\omega}_3}^2+{\overline{\omega}_4}^2)\ \cos(\overline{\phi})\ \cos(\overline{\theta})\ \Delta\phi \label{eq:model_y_transl} 
\end{flalign} 
Laplace transforming \autoref{eq:model_x_transl} and \ref{eq:model_y_transl} yields:
%
\begin{flalign}
    m\ \dot{x}_I(s)\ s&=-k_{th}\  (\omega_1 ^2 + \omega_2 ^2 + \omega_3 ^2 + \omega_4 ^2)\ \theta(s) \\
    m\ \dot{y}_I(s)\ s&= k_{th}\ (\omega_1 ^2 + \omega_2 ^2 + \omega_3 ^2 + \omega_4 ^2)\ \phi(s)
\end{flalign}
%
The inner velocity controller design requires a transfer function from each of the angles to each of the velocities, that can be written as:
%
\begin{flalign}
    G_{\dot{x}_I}(s)&=\frac{\dot{x}_I (s)}{\theta (s)}=\frac{-k_{th} (\omega_1 ^2 + \omega_2 ^2 + \omega_3 ^2 + \omega_4 ^2)}{m\ s} \label{eq:Gxdot} \\
    G_{\dot{y}_I}(s)&=\frac{\dot{y}_I (s)}{\phi (s)}=\frac{k_{th}(\omega_1 ^2 + \omega_2 ^2 + \omega_3 ^2 + \omega_4 ^2)}{m\ s}  \label{eq:Gydot}
\end{flalign}
%
\begin{where}
\va{G_{x_I}}{is the plant for the translational velocity in $x_I$ direction}{}
\va{G_{y_I}}{is the plant for the translational velocity in $y_I$ direction}{}
\end{where}

From \autoref{eq:Gxdot} and \ref{eq:Gydot} it can be noticed that the two plants are similar but with different sign. The controller design is carried out for the x translational velocity and applied to the both afterwards.

The systems is formed by an integrator and a gain. This means the systems are marginally stable in open-loop. $G_{x_I}$ has a negative gain, which means that it becomes unstable in closed-loop if no controller is placed. This feature requires a negative controller gain to compensate the plant.

It is noticeable that the control action for this design is the pitch angle, which means that there is an inner loop (the attitude control loop) that is able to produce the required angle in the system. To take these dynamics into account, an approximation of the behavior of the attitude controller can be done, using a second order system and looking at the step response when a pitch reference is applied.
%
\begin{figure}[H]
    \includegraphics[scale=.7]{figures/pitchResponseApprox}
    \centering			
    \captionof{figure}{Comparison between the behavior of the pitch due to the attitude controller and the approximation done with a second order system.} \label{fig:pitchResponseApprox}
   \end{figure} 
%
\begin{flalign}
        G_{pitch}(s)=\frac{\theta (s)}{\theta_{ref} (s)}=\frac{\omega^2_n}{s^2+\xi \ \omega_n \ s+\omega^2_n}=\frac{86.05}{s^2+16.14 \ s+86.05}
\end{flalign}
%
With this new dynamics added to the system, the design of the controller can be done. The final rot locus of the plant can be seen in \autoref{fig:rLocusVelocity}.
%
\begin{figure}[H]
    \includegraphics[scale=.7]{figures/rLocusVelocity}
    \centering			
    \captionof{figure}{Root locus of the plant for the velocity controller, that includes as well the dynamics of the attitude controller in pitch.} \label{fig:rLocusVelocity}
\end{figure} 
%
A proportional controller is considered sufficient, as there is an integrator in the plant so there is no need to compensate for a steady state error.

The value of this gain is chosen such that the system is as fast as possible without creating an overshoot and taking into account the control action required, which in this case is the pitch needed.

The final expression for the controllers can be written as \autoref{eq:Cxdot} and \ref{eq:Cydot}. Since the plant for the y controller has already a positive gain, there is no need to include a negative sign in the controller.
%
\begin{flalign}
C_{\dot{x}_I}(s)&= -0.19 \label{eq:Cxdot} \\
C_{\dot{y}_I}(s)&= 0.19 \label{eq:Cydot}
\end{flalign}
%
\begin{where}
    \va{C_{\dot{x}_I}}{is the controller for the translational velocity in $x_I$ direction}{}
    \va{C_{\dot{y}_I}}{is the controller for the translational velocity in $y_I$ direction}{}
\end{where}

The step responses and the corresponding control actions can be seen in \autoref{fig:stepVelocity} and \ref{fig:stepVelocityControlAction}.
%
\begin{minipage}{\linewidth}
    \begin{minipage}{0.5\linewidth}
        \begin{figure}[H]
            \includegraphics[scale=.55]{figures/stepVelocity}
            \centering
            \captionof{figure}{Step response from the reference to the translational velocity.}
            \label{fig:stepVelocity}
        \end{figure}
    \end{minipage}
    \hspace{0.03\linewidth}
    \begin{minipage}{0.5\linewidth}
        \begin{figure}[H]
            \includegraphics[scale=.55]{figures/stepVelocityControlAction}
            \centering
            \captionof{figure}{Required control action (pitch angle) to achieve the reference in translational velocity.}
            \label{fig:stepVelocityControlAction}
        \end{figure}
    \end{minipage}
\end{minipage}
%
Once the velocities controllers have been done, the position controllers can be designed following a similar procedure.

First, a transfer function from the velocity to the position is required, which is just an integrator.
%
\begin{flalign}
G_{x_I}(s)&=\frac{x_I (s)}{\dot{x}_I (s)}=\frac{1}{s}  \label{eq:Gx} \\
G_{y_I}(s)&=\frac{y_I (s)}{\dot{y}_I (s)}=\frac{1}{s}  \label{eq:Gy}
\end{flalign}
%
\begin{where}
    \va{G_{x_I}}{is the plant for the translational position in $x_I$ direction}{}
    \va{G_{y_I}}{is the plant for the translational position in $y_I$ direction}{}
\end{where}

As in the case of the velocity controller, the inner dynamics are taken into account in the design of the controller. For doing so, the dynamics of the velocity control loop (which includes the attitude controller) are added, which result in the root locus of \autoref{fig:rLocusPosition}.
%
\begin{figure}[H]
    \includegraphics[scale=.7]{figures/rLocusPosition}
    \centering			
    \captionof{figure}{Root locus of the plant for the position controller, that includes as well the dynamics of the inner control loop.} \label{fig:rLocusPosition}
\end{figure} 
%
As in the previous case, a proportional controller is enough to make the loop have the appropiate dynamics. The gain is then choose such that the system is as fast as possible and there is no overshoot.

The final expression for the controllers can be written as \autoref{eq:Cx} and \ref{eq:Cy}. Since the plant for the y controller has already a positive gain, there is no need to include a negative sign in the controller.
%
\begin{flalign}
    C_{x_I}(s)&= 0.8 \label{eq:Cx} \\
    C_{y_I}(s)&= 0.8 \label{eq:Cy}
\end{flalign}
%
\begin{where}
    \va{C_{x_I}}{is the controller for the translational position in $x_I$ direction}{}
    \va{C_{y_I}}{is the controller for the translational position in $y_I$ direction}{}
\end{where}

The step responses and the corresponding control actions can be seen in \autoref{fig:stepPosition} and \ref{fig:stepPositionControlAction}.

\begin{minipage}{\linewidth}
    \begin{minipage}{0.5\linewidth}
        \begin{figure}[H]
            \includegraphics[scale=.55]{figures/stepPosition}
            \centering
            \captionof{figure}{Step response from the reference to the translational position.}
            \label{fig:stepPosition}
        \end{figure}
    \end{minipage}
    \hspace{0.03\linewidth}
    \begin{minipage}{0.5\linewidth}
        \begin{figure}[H] 
            \includegraphics[scale=.55]{figures/stepPositionControlAction}
            \centering
            \captionof{figure}{Required control action (velocity) to achieve the reference in translational position.}
            \label{fig:stepPositionControlAction}
        \end{figure}
    \end{minipage}
\end{minipage}
%


%To determine how large the absolute gain can be without making the system unstable due to saturation issues, it is necessary to consider the bandwidth. \\ \\
%The data from the Vicon room is transmitted with 100 Hz. This means the attitude controller must run with 50 Hz as maximum to ensure the controller is slower than the sensor data. A rule of thumb states that the bandwidth of the system shall be 25 times smaller than the attitude controller. The desired bandwidth of the translational roll controller is calculated as follows:
%\begin{align}
%BW=2\  \pi\  \frac{f_s}{25}=2\  \pi \frac{50}{25}=12.57\label{eq:bw_X}
%\end{align}
%\begin{where}
%\va{BW}{is the bandwidth of the plant}{rad \cdot s^{-1}}
%\va{f_s}{is the sampling frequency of the plant}{Hz}
%\end{where}
%
%From \autoref{eq:bw_X} it is known, that the ideal bandwidth of the system is 12.57 rad/s. 
%\begin{figure}[H]
%	\centering
%	\includegraphics[width=0.7\textwidth]{figures/bode_x.png}
%	\caption{Bodeplot of the plant, with the bandwidth of 12.6 rad/s displayed.}\label{fig:bode_x}
%\end{figure}
%The bodeplot in \autoref{fig:bode_x} reveals that the magnitude is -2.15 dB at 12.6 rad/s and must be lowered by 0.85 dB. 

%The gain of the P-controller is found to be: 
%\begin{align}
%C_{x,y}=10^{\frac{-0.85}{20}}=0.907\\
%\end{align}

%The step response for the designed controller can be seen in \autoref{fig:step_x}
%\begin{figure}[H]
%	\centering
%	\includegraphics[width=0.8\textwidth]{figures/step_x.png}
%	\caption{Step response for P-controller with a gain of -0.907 and 0.907 for x and y respectively.}\label{fig:step_x}
%\end{figure}
%When examining \autoref{fig:step_x}it is seen that, as expected,there is no steady state error. The system settles at 0.41 seconds and has a rise time of 0.12 seconds. Both settling time and rise time is within the requirements \fxnote{we have not set the requirements - i just wrote it, it seems reasonable, but at some point we must set requirements, that are argued for in a previous chater that is to be written}. As it is a first order system, there is no overshoot and it can therefore be concluded that the P-controller meets all requirements.
%\\
%\\
%Before it can be implemented and tested on the quadcopter, it needs to be discretized. As it is P-controllers, it simply means to encounter the sampling rate. The discretized controllers are presented along with the continuous controller in \fxnote{make graph}.
%\\ \\
  
%\subsubsection*{Pitch translational controller}
%The roll and pitch translational models are very similar and will therefore be designed similarly. The design procedure will be the same as for the translational roll controller.\\

% 
%The desired bandwidth is 12,57 rad/s as derived in \autoref{eq:bw_X}. 
%\begin{figure}[H]
%	\centering
%	\includegraphics[width=0.7\textwidth]{figures/bode_y.png}
%	\caption{Bode plot of the plant with the 12.6 bandwidth displayed.}\label{fig:bode_y}
%\end{figure}
%The bode plot in \autoref{fig:bode_z} reveals that the magnitude is -8.2 dB at a magnitude of 12.6 rad/s. The magnitude has to be lifted by 5,2 dB to obtain the desired bandwidth. 




%Before designing the controller limit checks of the transfer function to verify if the model behaves as the plant is expected to in reality is carried out. \\
%\fxfatal{I am having trouble thinking about it, as it is not entirely intuitive to me, when the input is an angle and not a force - however if possible, i think we should have a short piece of text to show we have been critical to the math we have derived - to check it before continuing.} 
%
%Now that the limit checks confirms the sanity of the model, the controller can be designed. \fxnote{I still think that all of the above in this section shall be moved to the model chapter as conclusion of the chapter.}\\
%

%
%FOR NIELS : 
%
%\begin{figure}[H]
%	\centering
%	\includegraphics[width=0.7\textwidth]{figures/bode_z.png}
%	\caption{Bode plot of the plant with the 12.6 bandwidth displayed.}\label{fig:bode_z}
%\end{figure}
%The bode plot in \autoref{fig:bode_z} reveals that the magnitude is -8.2 dB at a magnitude of 12.6 rad/s. The magnitude has to be lifted by 5,2 dB to obtain the desired bandwidth. 
%
%The gain of the P-controller is found to be:
%\begin{align}
%C_y=10^{\frac{5.2}{20}}=1.82
%\end{align}
%The step response can be seen in

%
%The model expression for pitch is previously derived to be:
%\begin{align}
%m\Delta\ddot{y}_I = k_{th}({\overline{\omega}_1}^2+{\overline{\omega}_2}^2+{\overline{\omega}_3}^2+{\overline{\omega}_4}^2)\cos(\overline{\phi})\cos(\overline{\theta})\Delta\phi
%\label{eq:model_y_transl}
%\end{align}
%Laplace transforming \autoref{eq:model_y_transl_y} yields:
%\begin{align}
%m\  y_1(s)\  s^2= k_{th}\  (\omega_1 ^2 + \omega_2 ^2 + \omega_3 ^2 + \omega_4 ^2)\  \phi
%\end{align}
%The transfer function for the pitch is as follows:
%\begin{align}
%H_{y1}(s)=\frac{y_1(s)\  s}{\phi}=\frac{k_{th}\  (\omega_1 ^2 + \omega_2 ^2 + \omega_3 ^2 + \omega_4 ^2)}{m\cdot s}
%\end{align}
%\begin{where}
%\va{H_{y1}}{is the plant for the translational pitch}{1}
%\end{where}
