
The Linear Quadratic Regulator (LQR) approach is utilized to design an optimal state feedback for the controller. This approach is utilized together with Bryson's rule, which yields the possibility to prioritize the convergences of the states in the system. Furthermore it does this while being able to set boundaries for the control action applied to the system.
%In order to find the values of the feedback matrix, a Linear Quadratic Regulator (LQR) and Bryson's rule in order to find the optimal control and having the possibility of prioritizing the convergence of the different states while setting boundaries to the control action applied to the system. 

The LQR finds the state feedback that minimizes the cost function \autoref{eq:costfunction}. \cite{ssReference}. %seen in it is possible to find the necessary state feedback gains which minimizes the cost function in 
\begin{flalign} 
	J &= \int_{0}^{\infty} \vec{x}^T \vec{Q} \vec{x} + \vec{u}^T \vec{R} \vec{u} \ dt
     \label{eq:costfunction}
\end{flalign}
\begin{where}
	\va{\vec{Q}}{is the state-cost matrix}{}
	\va{\vec{R}}{is the input-cost matrix}{}
\end{where}

In the cost function, $\vec{Q}$ and $\vec{R}$ are positive semi-definite and positive definite matrices, respectively. $\vec{Q}$ defines the penalties for the error in the states making the cost function value higher for the prioritized states. $\vec{R}$ sets the weights for the inputs of the system such that the control action does not exceed its maximum limits. \cite{ssReference} 

For the LQR to minimize the cost function, the matrices $\vec{Q}$ and $\vec{R}$ need to be found. This is done using Bryson's rule as seen in \autoref{eq:weightingmatrices}. The $\vec{Q}$ and $\vec{R}$ matrices are diagonal matrices. The $\vec{Q}$ matrix elements are a function of the maximum acceptable value for the error in the states. The $\vec{R}$ elements set limits for the control action in each of the inputs. \cite{OptimalControlChristoffer}.
\begin{flalign} 
	Q_{ii} &= \frac{1}{\text{maximum acceptable value of }[x^2_i]}\\
	R_{ii} &= \frac{1}{\text{maximum acceptable value of }[u^2_i]}
	\label{eq:weightingmatrices}
\end{flalign}
The maximum state error and maximum control action work as soft boundaries that give priorities to the different states and control actions.% The final values are found by solving the cost function, here a compromise between a controller that reduces the errors fast and uses the least amount of control action is found.

The state feedback gain, $\vec{F}_{\mathrm{e}}$, can be calculated using the expression seen in \autoref{eq:optimalF}.
\begin{flalign} 
	\vec{F}_{\mathrm{e}} &= -\vec{R}^{-1}{\vec{B}_{\mathrm{e}}}^T\vec{P}
	\label{eq:optimalF}
\end{flalign}
\begin{where}
	\va{\vec{P}}{is the state-transfer matrix.}{}
\end{where}

The positive definite matrix $\vec{P}$, provides, through \autoref{eq:optimalF}, the feedback matrix which minimizes the cost function in \autoref{eq:costfunction}. This can be found using the Algebraic Riccatti equation \autoref{eq:optimalP}. \cite{OptimalControlChristoffer}
%
\begin{flalign} 
	{\vec{A}{\mathrm{e}}}^T\vec{P}+\vec{P}\vec{A}_{\mathrm{e}}-\vec{P}\vec{B_{\mathrm{e}}}\vec{R}^{-1}\vec{B}_{\mathrm{e}}^T\vec{P}+\vec{Q} &= \vec{0}
	\label{eq:optimalP}
\end{flalign}
%
Once $\vec{F}_{\mathrm{e}}$ is obtained, it can be split into $\vec{F}$ and $\vec{F}_{\mathrm{Int}}$ as the first six columns are the state feedback and the last three are the integral control. These matrices can be seen in \autoref{app:matricesSS}. In this way, the controller can be implemented as shown in \autoref{fig:DetailedControllerColorDiagram}. \cite{ssReference}


%This cost function contains weighting factors, the $\vec{Q}$ and $\vec{R}$ matrices. The weighting factors are used to either penalize or priorities different states and inputs. Thus yield a trade-off between states and control inputs. When weighting the states, it is possible to give the yaw more gain than the roll and pitch if necessary. By weighting the inputs, it is possible to ensure that they do not yield a control action which exceeds its limits. \cite{ssReference} 
%The $\vec{Q}$ matrix is semi-definite as the weighted states, $\vec{x}^T\vec{Q}\vec{x}$, has the possibility of becoming zero even though the states are above zero??? The $\vec{R}$ matrix is positive definite as an control action is always desired???
%By minimize the cost function, it is possible to regulate the $\vec{x}$ to zero as time goes to infinity??

%The weighting matrices, $\vec{Q}$ and $\vec{R}$, are then found through an iterations process, where a compromise between desired system performance and control input can be found. \cite{OptimalControlChristoffer} 

