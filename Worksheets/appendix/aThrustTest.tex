\chapter{Thrust and Motor Speed Relation}\label{app:potentiometerLin} 
\textbf{Name: Group 733}\\
\textbf{Date: 30/09 - 2016}

\subsubsection{Purpose}
Finding the relation between the speed of the motor and the thrust force generated by the propeller.

\subsubsection{Setup}
%\begin{figure}[H]
%	\centering
%	\includegraphics[scale=1]{figures/LabSetupLinearityTest.pdf}
%	\caption{Setup diagram}
%	\label{LabSetupRangeTest}
%\end{figure}\vspace{-5mm}

\subsubsection{List of Equipment}
\begin{table}[H]
	\begin{tabular}{|l|l|p{4.3cm}|}
		\hline%------------------------------------------------------------------------------------------------------------
		\textbf{Instrument}                                  &  \textbf{AAU-no.}  &  \textbf{Type}                       \\
		\hline%------------------------------------------------------------------------------------------------------------
		Tachometer                                           &  08246           &  Shimpo DT-205		                   \\
		\hline%------------------------------------------------------------------------------------------------------------
	    Power Supply (11.1 V) &  64565                   &  ES 030-5                 \\
		\hline%------------------------------------------------------------------------------------------------------------
		Processing Unit                                   &  N/A               & Arduino Mega     \\
		\hline%------------------------------------------------------------------------------------------------------------
		Motor                                   &  N/A               & Multistar 2213-935     \\
		\hline%------------------------------------------------------------------------------------------------------------
		Motor Speed Controller                                   &  N/A               &  -      \\
		\hline%------------------------------------------------------------------------------------------------------------
		Propeller                                   &  N/A               & Turnigy 1045R     \\
		\hline%------------------------------------------------------------------------------------------------------------
		
	\end{tabular}
\end{table}

\subsubsection{Procedure}
\begin{enumerate}
	\item Construct the setup as seen in figure EXPFIGURE, the power supply is connected to the motor driver and the Arduino Mega is powered from the computer. One PWM pin and GND pin from the board must be connected to the driver signal cables yellow and brown respectively. 
	\item Start the program with a fixed duty cycle.
	\item Wait for the speed to stabilize and read the scale value.
	\item Measure the rotational speed with the tachometer.
\end{enumerate}


\subsubsection{Results}
\begin{table}[H]
	\centering
	\begin{tabular}{|l|l|p{4.3cm}|}
		\hline%------------------------------------------------------------------------------------------------------------
		\textbf{Rotational Speed [rpm]}    &  \textbf{Thrust Force [g]}          \\ 
		\hline%------------------------------------------------------------------------------------------------------------
		2240                                         & 69               \\
		\hline%------------------------------------------------------------------------------------------------------------
		2305 										  & 74               \\
		\hline%------------------------------------------------------------------------------------------------------------
		2445                              			  & 85               \\
		\hline%------------------------------------------------------------------------------------------------------------
		2495                              			  & 89               \\
		\hline%------------------------------------------------------------------------------------------------------------
		2585                                         & 97               \\
		\hline%------------------------------------------------------------------------------------------------------------
		2665 										  & 99               \\
		\hline%------------------------------------------------------------------------------------------------------------
		2811                              			  & 114               \\
		\hline%------------------------------------------------------------------------------------------------------------
		2995                                          & 129              \\
		\hline%------------------------------------------------------------------------------------------------------------
		3195 										  & 150               \\
		\hline%------------------------------------------------------------------------------------------------------------
		3287                              			  & 160               \\
		\hline%------------------------------------------------------------------------------------------------------------
		3493                                          & 182               \\
		\hline%------------------------------------------------------------------------------------------------------------
		3609 										  & 195               \\
		\hline%------------------------------------------------------------------------------------------------------------
		3765 										  & 215               \\
		\hline%------------------------------------------------------------------------------------------------------------
		3888 										  & 228               \\
		\hline%------------------------------------------------------------------------------------------------------------
		4060 										  & 250               \\
		\hline%------------------------------------------------------------------------------------------------------------
				
	\end{tabular}
\end{table}


\subsubsection{Results from Linearity Test}
Result of the test shows that below \si{-39,5^{\circ}} the potentiometer has a dead area. The dead area might come from the continuous rotation of the potentiometer, since the measurement are very near to this point where the potentiometer changes. The area have at dead span from \si{5^{\circ}} to \si{10^{\circ}}.

The graph shows the measured values according to angle.

%\begin{figure}[H] 
%	\centering 
%	\includegraphics[scale=0.7]{figures/linearityOfPotmeterTest2-1}
%	\caption{Result from linearity test}
%	\label{linearityOfPotmeterTest2-1}
%\end{figure}
Because of the dead area the potentiometer could be rotated so the frame will be turning in this area, but since the Cubli has been built like this and the code has some hardcoded value of the potentiometer and the area are not used then it will be left as it is. Also because the software is distributed on different machines it has to be changed on every system.


\subsubsection{Results of Equilibrium Zone}
During the test the equilibrium has varied, and area where it can stand balanced have been measured. 

\begin{table}[H]
	\centering
	\begin{tabular}{|l|l|p{4.3cm}|}
		\hline%------------------------------------------------------------------------------------------------------------
		\textbf{Equilibrium range in degrees}       &  \textbf{mV}         \\
		\hline%------------------------------------------------------------------------------------------------------------
		-0,44                               			  & 211,80               \\
		\hline%------------------------------------------------------------------------------------------------------------
		-0,05                                          & 213,64               \\
		\hline%------------------------------------------------------------------------------------------------------------
		0,053 										  & 217,00              \\
		\hline%------------------------------------------------------------------------------------------------------------
	\end{tabular}
\end{table}

%The graph shows the measured values according to angle of equilibrium area.
%\begin{figure}[H] 
%	\centering 
%	\includegraphics[scale=0.7]{figures/linearityOfPotmeterTest2-2}
%	\caption{Raw test data plot}
%	\label{linearityOfPotmeterTest2-2}
%\end{figure}

Since the frame is connected to the baseplate through the potentiometer and this one is kept in place by bearings, the only force keeping the frame standing is the friction between them and potentiometer. This region is about \si{1^{\circ}}.


