\subsection{Controller for z axis}
%Derivation of the transfer function 
%
%Root locus and bode
%
%Problem with P controller (Input disturbance)
%
%Simulation of P controller
%
%New controller 
%
%simulation of New controller

It is decided to control the velocity in the z-direction in the inertial system. In order to design such a controller, the transfer function is obtained. The four velocities motor velocities, as input and the velocity in the z-direction in the inertial system, as output.
From \autoref{eq:FinalLinearEquationZ}, with the corresponding translational linearized block diagram in \autoref{fig:TranslationalLinearModelBlockDiagram}, the linear transfer function for the z-direction is readily obtained.
%
\begin{flalign}
  \frac{\dot{Z}_I}{\omega_{sum}} &= \frac{ \frac{1}{4}\ (-2 k_{th})\ \overline{\omega}_{sum} }{ m\ s } & \label{eq:linearTransferFunctionZ}
\end{flalign}

\begin{where}
  \va{\dot{Z}_I}{is the velocity in the z-direction in the inertial system}{}
  \va{\omega_{sum}}{is the sum of velocities to be controlled}{}
  \va{\overline{\omega}_{sum}}{is the sum of rotor velocities in equilibrium}{}
  \va{k_{th}}{is the thrust constant}{}
  \va{m}{is the mass of the quadcopter}{}
\end{where}

This system has a pole in zero and a negative gain, which means that the locus on increasing gain will drive the system into the right half plane and make it unstable. Thus a negative gain is applied as the initial P-controller. If a gain of $-200$ is applied, the controller will bring the velocity to \SI{1}{m \cdot s^{-1}} in \SI{2}{s}, see \fxnote{insert\autoref{fig:}}.

\fxnote{figure of response of P-controller using the transfer function}

However, the linearized transfer function in \autoref{eq:linearTransferFunctionZ} does not capture the relation to gravity, since it only describes the system in equilibrium. If the needed equilibrium speeds of the rotors are altered, then this linearized transfer function only sees it as an increase in gain, whereas the real system will see it as an input disturbance.
Thus, the P-controller is tested on the nonlinear model with an added difference of \SI{10}{rad \cdot s^{-1}} in the each of the 4 needed equilibrium speeds in the model. The simulation is seen below in \fxnote{insert\autoref{fig:}}, where the reference input is subjected to a step of \SI{1}{m \cdot s^{-1}}, however, the velocity stabilizes instead at \SI{0.9}{m \cdot s^{-1}}, revealing a steady state error.

\fxnote{figure of response of P-controller using the nonlinear model}

In order to remove the steady state error an integral is therefore introduced to the controller. The root locus of the system, which now contains two poles in zero, will branch along the imaginary axis. To avoid oscillations the two loci are attracted by use of a zero, placed on the left real axis as seen on \fxnote{\autoref{fig:}}.

\fxnote{figure of root locus with two integrators and a zero}

The zero is placed and the gain is scaled to achieve the same rise time as with the P-controller, \SI{2}{s}.

\fxnote{simulation with linearized transfer function?}
\fxnote{simulation in nonlinear model}

