%In the last years, the interest for quadcopters has increased due to the great possibilities they offer. Among these, the most well-known ones are surveillance, inspection of big structures and search and rescue missions in difficult environments.
%A design that is able to make the quadcopter hover and move to a desired position is designed. The system’s coupled behavior and instability raises a challenging control task.

%This task is solved by implementing a linear control design. The system is split into an attitude and translational model. These are controlled individually by state space and classical controllers respectively. The prototype gets its attitude and position from a motion tracking system based on infrared cameras, keeping the control in a micro processor on the quadcopter. This layout constitutes a distributed system, where network issues, such as delays and packet losses, are taken into account.
Quadcopters constitute a control challenge due to their inherent instability and coupled behavior. However, the interest for them has increased due to the multiple possibilities they offer. A linear control solution capable of stabilizing the quadcopter and controlling its position is presented by combining state space and classical control approaches. The presented results include the attitude control performance and simulations showing the behavior of the translational controllers.